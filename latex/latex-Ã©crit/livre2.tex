\part{De la nature de l'âme, de la présence divine et de ses lois}

\chapter{Sur l’absence de silence de Dieu}

Là où un panthéiste pourrait considérer observer Dieu en toute chose, d'autres, comme moi, considèrent que le matériel et l'immatériel sont distincts de Dieu et résultent de sa création.

Dans ces deux cas, il est toujours possible d'acquérir des connaissances sur le divin (ou, dans le cas d'un athée, sur un univers émanant de lui-même, dépossédé de sens).

Si Dieu était totalement silencieux, nous n'observerions ni matière, ni n'entendrions de sons, et ainsi de suite. Nous serions totalement coupés de son univers. Un silence total de Dieu impliquerait que nous soyons également coupés de nous-mêmes, et il nous faudrait alors être dépourvus de conscience de nous-mêmes et de notre environnement. La simple observation et les interprétations de l'univers renseignent alors sur les mystères du divin.

Pour autant, religions et sciences, sur des plages temporelles millénaires, ont toujours évolué vers plus de justesse puisque l'homme, au fil des âges, a toujours pu acquérir une compréhension plus profonde de son environnement et des lois qui le régissent.

Au moins un indicateur incontestable est l'évolution de ces sciences et l'élaboration d'outils et de machines de plus en plus sophistiquées, permettant une meilleure maîtrise de notre environnement. Ces outils, qui à leur tour (par exemple la lentille convergente), ont permis, par leur emploi, d'augmenter notre savoir, d'enrichir encore notre compréhension du monde, d'élaborer de nouveaux outils, et ainsi de suite.

La création de ces outils, résultant en premier lieu d'une observation des lois de l'univers, a été permise par des évolutions et des bouleversements paradigmatiques, métaphysiques et des recherches fondamentales. La construction de machines n'étant qu'une discipline de l'ingénierie s'appuyant sur des recherches savantes abstraites (et, en premier degré, la compréhension de la création divine) donne l'impression d'être tout à fait inutile lorsqu'il s'agit de sciences si elles ne permettent pas la construction d'objets utiles.

Nous pouvons émettre le même raisonnement concernant les découvertes en termes d'éthique et de mœurs toujours plus adaptées et améliorant le bien-être humain. Cependant, pour un athée, il me semble que ces choses sont de pure abstraction qui n'ont pas de sens concret.

En résumé, Dieu ou l'univers nous communique ses savoirs en nous permettant l'observation, l'interprétation, la réflexion, et la construction matérielle et spirituelle. Dieu est avant tout une recherche humaine qui a certainement débuté dès son apparition. Cette recherche est donc millénaire et ses avancées sont très lentes. Nous n'apprendrons sans doute pas grand-chose de plus de notre vivant, mais ses mystères se dévoilent siècle après siècle. Dieu n'est donc pas silencieux car il nous autorise à minima à observer sa création et à la comprendre.



\chapter{Perception religieuse de la nature de l'univers}

Pour un individu déiste, tout comme pour un individu athée, le constat le plus élémentaire que l’on puisse faire vis-à-vis de notre rapport à l’univers est que nous sommes tous soumis, collectivement comme individuellement, à un ensemble de lois et de contraintes dues à notre environnement.

Les lois fondamentales de la physique ne peuvent être violées et, à titre d’exemple, on ne peut pas décider d’atterrir sur la Lune par un simple saut en hauteur. Il faut s’équiper du matériel adéquat.

D’autres lois, pourtant plus permissives, peuvent être contrées par l’exercice de notre libre arbitre. Ces lois ont, il me semble, une existence bien concrète, puisque autoriser le meurtre et le vol, par exemple, serait une catastrophe pour notre vie collective. Si ces crimes se généralisaient pour ne plus être réduits à des épiphénomènes, alors notre environnement ne serait plus propice à notre simple survie.

On peut dans ce cas, faire le constat que certaines actions humaines sont plus adaptées que d’autres pour la vitalité d’une civilisation et des individus qui la composent. Le tout reste alors de comprendre quelles sont les meilleures règles à adopter, d’abord pour notre survie, puis pour notre bien-être.

Pour certains, il s’agit de lois naturelles, et on peut souvent entendre que ces choses proviennent de "la nature". Pour ma part, il s’agit là d’une vision réductrice de cet ensemble de phénomènes, car je ne peux pas m’empêcher de m’interroger sur la nature de cette "nature". D’où et de quoi proviennent ces lois ? Qu’est-ce qui les maintient ? Comment ont-elles été créées et pour quelle(s) raison(s) ? Comme pour la physique, ne s'agirait-il pas de lois divines ?

Il est toujours possible de songer que le monde, l’univers dans lequel nous baignons, n’est qu’une illusion, est irréel, et n’a aucune existence concrète. Certains contemporains le pensent et écrivent sur ce sujet, mais cette interprétation figure aussi dans des croyances beaucoup plus anciennes, comme le maya des hindouistes et des bouddhistes. Cependant, une illusion doit être causée par au moins un élément ayant une existence concrète. Si cet élément disparaît, il en va de même pour l’illusion créée.

Il s’agit d’un raisonnement métaphysique simple qui dit que si absolument tout est irréel, alors tout est réellement irréel, et nous nous heurtons alors à un paradoxe. Cette ontologie religieuse est une des différences majeures avec celles apportées par les religions helléniques et judéo-chrétiennes qui, quant à elles, admettent une création et une existence concrète de l’univers, tandis que les religions issues du brahmanisme admettent un univers irréel produit par le rêve de Dieu.

Il n’y a alors que trois solutions :

\begin{itemize}
\item du point de vue de l'athéisme occidental, l’univers a un jour émané du néant et évolué pour prendre la forme dans laquelle nous le trouvons aujourd’hui.
\item l’univers peut aussi être potentiellement imaginé ou rêvé par quelque chose qui ne peut alors qu’être une entité pensante. Soit l’univers n’existe pas concrètement.
\item Il est enfin possible que l’univers ait été créé ou engendré par une entité et nous vivons dans un environnement matériel réel qui a nécessairement été pensé et créé.
\end{itemize}

Étant convaincu que nous possédons un libre arbitre, j’adhère personnellement à la troisième hypothèse puisque si l’univers est imaginé, les actions humaines le sont aussi, et nous vivons alors dans l’illusion de posséder un libre arbitre. De plus, je considère comme tout à fait juste le célèbre principe de René Descartes : \begin{quote}\textit{"Je pense, donc je suis"}\end{quote}, qui implique alors que tout être pensant existe bel et bien en tant qu’étant. Si les êtres vivants étaient eux aussi imaginés, ceux-ci ne posséderaient donc pas la pensée. Enfin, nous avons vu au chapitre 4 que l'univers ne peut émerger du néant, et un principe créateur non engendré et donc non préalablement raisonné me semble aberrant.

Pourtant, nous n’observons et ne pensons pas l’univers tel qu’il est réellement, sinon nous en aurions tous une connaissance parfaite et toutes les sciences deviendraient inutiles. L’association de nos observations sensorielles et de notre capacité d’abstraction serait suffisante pour comprendre absolument tout.

Il faut alors s’appuyer sur le mythe de la caverne de Platon, très connu des académiciens. Voici donc un extrait du septième livre de "La République" de Platon:

\begin{quote}
\textit{"Eh bien, après cela, dis-je, compare notre nature, considérée sous l’angle de l’éducation et de l’absence d’éducation, à la situation suivante ; Représente-toi des hommes dans une sorte d’habitation souterraine en forme de caverne. Cette habitation possède une entrée disposée en longueur, remontant de bas en haut tout le long de la caverne vers la lumière. Les hommes sont dans cette grotte depuis l’enfance, les jambes et le cou ligotés de telle sorte qu’ils restent sur place et ne peuvent regarder que ce qui se trouve devant eux, incapables de tourner la tête à cause de leurs liens.
Représente-toi la lumière d’un feu qui brûle sur une hauteur loin derrière eux et, entre le feu et les hommes enchaînés, un chemin sur la hauteur, le long duquel tu peux voir l’élévation d’un petit mur, du genre de ces cloisons qu’on trouve chez les monteurs de marionnettes et qu’ils érigent pour les séparer des gens. par-dessus
ces cloisons, ils montrent leurs merveilles.}
\begin{itemize}
\item \textit{Je vois, dit-il.}
\item \textit{Imagine aussi, le long de ce muret, des hommes qui portent toutes sortes d’objets fabriqués qui dépassent le muret, des statues d’hommes et d’autres animaux, façonnés en pierre, en bois et en toute espèce de matériau. Parmi ces porteurs, c’est bien normal, certains parlent, d’autres se taisent.}
\item \textit{Tu décris là, dit-il, une image étrange et de bien étranges prisonniers.}
\item \textit{Ils sont semblables à nous, dis-je. Pour commencer, crois-tu en effet que de tels hommes auraient pu voir quoi que ce soit d’autre, d’eux-même et les uns des autres, si ce n’est les ombres qui se projettent, sous l’effet du feu, sur la paroi de la grotte en face d’eux?}
\item \textit{Comment auraient-ils pu, dit-il, puisqu’ils ont été forcés leur vie durant de garder la tête immobile?}
\item \textit{qu’en est-il des objets transportés? N’est-ce pas la même chose?}
\item \textit{Bien sûr que si.}
\item \textit{Alors, s’ils avaient la possibilité de discuter les uns avec les autres, n’es-tu pas d’avis qu’ils considéreraient comme des êtres réels les choses qu’ils voient?}
\item \textit{Si, nécessairement."}
\end{itemize}
\end{quote}

Cette allégorie vient nous expliquer que notre condition humaine implique que nous observons et interprétons l’univers sous une forme déformée de la réalité, partiellement illusoire. Notre prison et nos chaînes qui nous empêchent d’observer et de comprendre parfaitement la réalité sont les limites de nos sens, de notre capacité d’interprétation, et de la qualité de nos raisonnements.

Il reste pourtant possible de s’émanciper partiellement et peut-être (qui sait ?) un jour totalement de cette infirmité. Nous pouvons alors citer Joseph Fourier:

\begin{quote}
\begin{center}
\textit{"Les Mathématiques sont une faculté de la raison humaine, destinée à suppléer à la brièveté de la vie et à l’imperfection des sens."}
\end{center}
\end{quote}

J'ajouterai qu'on pourrait en dire autant de toutes les sciences, théologie et philosophie comprises. Si les sciences permettaient un jour d’atteindre la compréhension et la maîtrise totales des phénomènes naturels, serait-ce réellement sage ? Je ne crois pas que l’homme soit assez mature pour tout contrôler avec une discipline infaillible. Une connaissance totale de l’univers implique certes une connaissance parfaite de l’éthique transcendante qui y serait à priori incluse, mais encore faut-il être en mesure de se l’imposer à soi-même de manière tout aussi parfaite. Ce serait une tâche herculéenne, même pour le plus vertueux des hommes, de par notre nature pécheresse.

Un chrétien convaincu vit aussi dans la conviction qu’à la fin des temps, tout sera révélé. La connaissance et la maîtrise de toute chose feraient pourtant de nous des individus omnipotents, et pour un chrétien, il ne peut y avoir plusieurs dieux. Une révélation totale implique alors que l’individu, ne pouvant devenir lui-même un dieu, fait alors partie intégrante du Dieu avec lequel il ne fait qu’un.

Bien que les théologiens du christianisme ne le pensent pas tous de la sorte, c'est le cas dans le catholicisme contemporain, et il est aussi possible d'y voir une analogie avec la pensée hindoue où, passé un certain nombre de réincarnations successives, les âmes retournent à l’âme primordiale de laquelle elles sont issues. Parfois, les religions s’influencent mutuellement. Il est toujours possible qu’elles soient toutes issues et dérivées d’une première et unique religion primitive. Après tout, au XIXe siècle, les historiens considéraient toujours un monothéisme originel qui s'est dénaturé après la chute de Babel et l'apparition du zoroastrisme, la religion la plus ancienne encore pratiquée aujourd'hui. Ce n'est plus une pensée dominante à notre époque, mais elle compte toujours des adhérents.

Peut-être que l’hindouisme n’a nullement influencé le christianisme, le débat est toujours ouvert. Dans tous les cas, les solutions plausibles à ces questions métaphysiques sont en nombre assez limité.

\chapter{L'âme et la matière}

Définissons la conscience d'un étant par le fait qu'il possède la connaissance de sa propre existence et, dans la plupart des cas, de ce qu'il perçoit comme étant son environnement de façon dissociée.

Une question importante, toujours en vigueur à notre époque, est celle de l'âme ou de la conscience engendrée ou non par la matière. En d'autres termes, une machine peut-elle être rendue consciente ? Sommes-nous nous-mêmes semblables à des machines conscientes grâce à une activité cérébrale qui n'est, en somme, que des électrons en mouvement ?

Pour le philosophe René Descartes, de par sa notion de dualisme, le corps de l'homme est semblable en tous points à une machine physique, mais sa pensée, sa capacité d'introspection sur ses raisonnements l'en différencient. Son âme est métaphysique.

Certains penseurs comme Karl Marx ou Thomas Hobbes ont émis l'assertion que la conscience émane directement de la matière en mouvement. Dans ce cas, il serait un jour possible d’insuffler une âme à une machine.

Cette question revient actuellement sur le devant de la scène, jusqu'à atteindre le grand public, avec les récentes avancées en matière d'intelligence artificielle. Pour autant, ces algorithmes simulant une forme d’intelligence ne sont en rien conscients. D'ailleurs, un ordinateur contrôlé par n'importe quel programme est aussi, en premier lieu, une mise en mouvement d'électrons dans des circuits.

Je réfute personnellement l'idée que la conscience puisse émaner de matière en mouvement, même complexe (si tant est que la nature comprenne la notion de complexité). Cela signifierait que le mouvement de matière, purement matériel par tautologie, peut engendrer des étants immatériels, puisque la conscience existe en tant qu'immatériel. Il manquerait donc des connaissances fondamentales sur la nature de la matière, car au vu des connaissances scientifiques contemporaines où toutes les lois physiques régissant le mouvement de matière sont considérées comme parfaitement connues, ce phénomène est impossible.

De plus, la physique nous enseigne justement que rien ne se crée, et selon la maxime célèbre d'Antoine de Lavoisier : \begin{quote}\textit{"Rien ne se perd, rien ne se crée, tout se transforme."}\end{quote}. Un mouvement de matière, engendrant une conscience, le tout sans transfert et donc perte de matière pour alimenter cette conscience, viole la physique la plus élémentaire.

Donc, de mon point de vue, la conscience est un immatériel non engendré par la matière et son mouvement. Nous possédons par conséquent une âme. Puisqu'il ne s'agit pas de matière et qu'un lieu est défini par de la matière dans un espace, la conscience, ou l'âme, contrairement à notre intuition première qui nous pousse à la situer dans notre tête, ne posséderait à priori pas de lieu.

Le propre des étants immatériels reste, semble-t-il, leur incorruptibilité, c'est pourquoi je crois fermement que toute conscience humaine ou animale survit à la mort de son enveloppe charnelle. Donc, pour ma part, l'âme doit être immortelle.

Cependant, je pense que ces mouvements d'électrons caractérisant l'activité cérébrale ont une utilité, car, d'une part, nous avons vu dans le chapitre 5 que rien n'est créé sans raison. Je subodore qu'à l'instar d'un supercalculateur dont la technologie nous dépasse encore, cette activité cérébrale est interprétée par notre conscience en ce qui concerne nos sens mais aussi nos facultés cognitives.

Puisque nous percevons ces éléments, il semblerait que, à minima, le long de notre vie terrestre, ce que nous sommes sur le plan matériel communique et interagit avec notre étant immatériel pour acquérir expérience et connaissance. L'inverse est aussi valable, puisque notre conscience nous pousse à exécuter des actions et à utiliser nos facultés cognitives. C'est de cette manière que le monde matériel laisse une empreinte sur notre âme, et que notre âme, lors de notre passage sur terre, laisse aussi son empreinte au monde matériel. Ceci est mon hypothèse.

\chapter{De la notion d'éthique dans la religion}

L'aspect éthique d'un individu découle d'ensembles de notions vues comme absolues. En effet, l’éthique ne peut pas être vue comme une notion absolue en elle-même, car elle nécessite de faire intervenir d’autres vérités plus élevées et est également confrontée aux perceptions émotives des individus. Le questionnement éthique et son établissement permettent d’obtenir une certaine conviction de discernement entre le bien et le mal.

C’est donc que les croyances de l’individu influent grandement sur son sens éthique. On peut déjà le constater dans les différentes mœurs qu'ont connues les peuples à travers l’histoire. Ces mœurs ont été forcément vues comme éthiques par ceux qui les pratiquaient et parfois comme abjectes par les peuples voisins. Pour un peuple occidental moderne, les guerres saintes (non exclusivement celles menées par l’Islam) sont vues comme des actions allant à l’encontre de leurs valeurs. Il en est bien sûr de même pour des notions encore plus barbares telles que les sacrifices humains rituels ou le cannibalisme. Dans le cadre de l'athéisme occidental, ce sont les valeurs humanistes qui sont considérées comme fondamentales et sur lesquelles l'éthique de ces sociétés est fondée.

\begin{center}
***
\end{center}

L’éthique d’un individu dépendra donc de son éducation religieuse (puisque nous considérons ici l'athéisme comme une religion) et de ses émotions et expériences propres. Son éducation a pour effet de modeler en partie ses émotions, tandis que ses émotions modèlent la perception qu’il a de son éducation et/ou de son attrait au divin.

On voit aussi que, dans le cas de l’éthique, on peut effectuer une dichotomie entre une éthique provenant d’une considération du divin, donc au-dessus des hommes (considérée comme transcendante pour le croyant), et une éthique théorisée par des penseurs humains sans lien avec le religieux. Dans le premier cas, l’individu n’ira (sauf en cas de reconnaissance d’un prophète) ni réinventer ni contester l’éthique qui lui a été enseignée. Dans le second cas, cette éthique peut être contestée, réinventée, remodelée, et cela, même à grande vitesse, lorsque la personne qui en a été instruite en éprouve le besoin ou la volonté. L’individu pourra aussi s’autoriser à réviser son éthique lorsqu’il estimera avoir dépassé son pédagogue.

Donc, dans le cas de religions basées sur des écrits millénaires, et bien que les interprétations des textes puissent être révisées, les évolutions en termes d'éthique sont beaucoup plus lentes que pour l'athéisme occidental.

Pour cette raison, dans le cas d’un individu ou d'un groupe athée, l’éthique peut subir des changements beaucoup plus brusques et rapides, car les notions les plus générales de leur éthique, et dont les autres découlent, sont tout à fait disposées à être renversées. De plus, autant un groupe d’individus de la même religion partagent au minimum les valeurs les plus hautes d’une éthique commune, autant des individus athées ayant fait varier trop rapidement et profondément leur éthique au gré de leurs expériences de vie seront aussi plus fragilisés dans leurs liens et actions communautaires, lorsque chacun d’entre eux adopte une éthique tout à fait personnelle et parfois non conforme à celle de ses congénères ou à l'éthique humaniste. Il en est de même lorsque plusieurs religions cohabitent au sein d'une même nation et que des tensions religieuses finissent par émerger en partie à cause de divergences éthiques. D'ailleurs, l'humanisme est inclusif avec toutes les religions, cherche à les assimiler et les faire cohabiter, ce qui peut constituer un potentiel danger.

\chapter{Des actions individuelles aux actions collectives}

Tandis que les religions primitives ne traitaient que de phénomènes collectifs sans s'attacher aux actions individuelles, depuis le judaïsme (et sans doute dans des religions plus anciennes), un Dieu caractérisant à la fois le personnel et le collectif est considéré. C'est à ce moment non daté de l'histoire qu'a émergé l'idée du caractère individuel et collectif des hommes.

Dans le chapitre précédent, nous avons traité l'éthique principalement sous l'angle des actions individuelles. Il est pourtant suffisamment simple de constater qu’une action individuelle se répercute sur le plan collectif dès qu’elle est observée. Le meurtre d’un individu atteint aussi la personne tuée et, par extension, ses proches ou une plus large partie de la population si l’acte est plus massivement diffusé. Il en est de même pour une action neutre ou bénéfique pour l’individu qui la reçoit. Les moindres échanges commerciaux peuvent même influencer l’économie à plus ou moins grande échelle.

Chaque action d’un individu unique sera donc perçue comme positive, neutre ou négative, éthique ou contraire à l'éthique, par chaque personne affectée. Cette action isolée influe donc à plus ou moins grande échelle dans la sphère collective.

Ces individus marqués par cette action peuvent choisir d’agir de manière passive, c’est-à-dire a minima, analyser avec plus ou moins d’intérêt les raisons pour lesquelles cette situation s’est déclenchée et ainsi enrichir une fois de plus leur système de croyances.

En revanche, lorsque l’individu estime que l’action est un vecteur ou une entrave suffisante à l’atteinte de ses désirs, celui-ci peut choisir de s’intégrer à l’action collective. De là, il essaiera par ses moyens, d’amoindrir, d’amplifier, d’annuler ou encore de faire dévier l’action en cours.

De ce fait, lors d’une action individuelle menant à une action collective, certains acteurs n’interviendront pas, tandis que d’autres interviendront par une prise de parti active.

On peut donc juger de l’ampleur d’une action collective par le nombre de personnes prenant part à cette action, mais aussi par les moyens qu’ont ces personnes pour faire pencher la victoire dans le camp souhaité. Les témoins passifs peuvent tout de même être pris en compte dans l’ampleur du phénomène car ils sont d’une part affectés par les évènements et d’autre part agissent par leur inaction.

On remarque aussi qu’un acteur passif n’agit que dans sa propre sphère individuelle et non pour l'intérêt collectif. Quant à l’action active, elle est efficiente à la fois dans la sphère individuelle et collective. Les réflexions qu’impliquent les actions passives ne sont pas toujours réalisées en vue d’une action active ultérieure, puisque l’individu peut choisir de rester systématiquement passif vis-à-vis d’une même action qui se répète ou de n'en tirer aucun enseignement. Aussi, l'action passive d’un individu reste tout de même une action active vue par les individus dans l’action, puisque refuser de prendre part à l'action en cours est une forme de contestation réduisant les effectifs nécessaires à la réalisation du projet en cours. Une action passive n’est donc réellement passive que pour l’individu qui l’exécute et résulte d’un simple choix de ne pas faire pencher la balance vis-à-vis de l’action qu’il observe.

Aucun phénomène collectif n’émerge directement, c'est le fruit d’une action en premier lieu individuelle et dont le résultat aboutit à la prise de parti d’un groupe de personnes jugeant du bien-fondé de cette action de manière personnelle et collective. Il en résulte tout de même que chaque individu, bien qu’affecté par les actions du groupe, effectue lui-même le choix personnel d’enrichir ou non en effectifs et en moyens l’action en cours.

Bien sûr, dans certains cas, l’individu peut être menacé s’il ne prend pas part à l’action, mais aura toujours le choix de subir les châtiments dictés par les menaces plutôt que de prendre part au projet. Dans le cas d’une guerre, les menaces sont cependant suffisamment importantes pour empêcher en grande partie la désertion.

\begin{center}
***
\end{center}

De ce constat, une action collective est le résultat d’un ensemble d’actions déclenchées par une action individuelle unique. Toute action individuelle ou collective a lieu par l'influence d'action(s) antérieure(s) effectuée(s) par des personnes tierces, hormis les actions destinées à sustenter nos besoins vitaux ou visant à l'auto-satisfaction. Ce serait donc un chaos total sans communication adaptée, une dimension éthique commune entre les individus exécutant l’action, et des règles adéquates et préalables à respecter.

La règle unique : « attrapez ce chat » pourrait prendre un aspect risible si aucun des trois prérequis cités plus haut n’étaient respectés. La dimension éthique dira par exemple si on a la possibilité de tuer le chat en l’attrapant ou de lui briser une patte. Les règles préalables éviteront des actions en désaccord et inharmonieuses entre les participants, tandis qu’une communication efficace permettra des mouvements plus fluides dans la réalisation des actions permettant l'aboutissement du projet.

Que ce soit pour le cas des guerres présenté plus haut ou pour un autre dessein, sans grammaire régissant les actions collectives, toutes les actions humaines collectives comme individuelles deviendraient inharmonieuses et inefficaces. C'est pourquoi une importance capitale est donnée à la discipline militaire, toute défaite ayant dans ce cas des conséquences dramatiques.

L'éthique commune, l'ensemble de règles communes ou particulières, une communication adaptée, et une hiérarchie pertinente sont donc les quatre piliers pour une organisation cohérente menant à la réussite de n'importe quel projet collectif.

Dans le livre I, nous avons mis en avant une source d'énergie originelle, puis une mise en mouvement originelle. Mais par ce raisonnement, on peut aussi admettre une action consciente originelle ayant inspiré toutes les autres. Qui en est responsable et quand la dater ? N'est-ce pas avant tout le résultat de l’œuvre de l'être suprême à l'origine de la création?

\begin{center}
***
\end{center}

Pour ces raisons, dans le cas d'une civilisation saine, composée d’un nombre conséquent d’individus, il y a nécessité de créer un système de lois morales et juridiques évitant au mieux notamment les actions collectives et individuelles violentes, et permettant une réalisation facilitée des desseins estimés comme essentiels à cette société, le tout permettant de nettement amoindrir toutes les actions représentant un danger au niveau collectif afin d'assurer la pérennité de cette civilisation.

Ces éléments, une fois présents, doivent (s’ils sont habilement conçus) permettre une prompte réalisation des desseins multiples d’un individu, d’un groupe d’individus, d’une société, d’une civilisation en faveur du groupement humain composé.

Pour résumer, les actions personnelles ou communes doivent favoriser le développement des sociétés au même titre que la société favorise la survie et le bien-être des citoyens qui la composent. Tous ces desseins additionnés (si leurs entreprises ont réussi) alimentent donc un système qui les alimente en retour.

En revanche, il n’arrive certainement jamais que tous les individus de ce système partagent à parts égales les fruits de l’aboutissement des actions collectives.

Malgré tout, il y a nécessité de déterminer la grammaire (ou l'ensemble de règles) la plus adaptée possible pour qu’un individu ou un groupe parviennent à leurs fins. Le choix du terme "grammaire" prend appui dans la théologie où il est possible de voir ces actions comme des imitations imparfaites d’un langage pouvant être considéré comme parfait.

Il y aurait alors potentiellement existence mais méconnaissance de la grammaire (donc des règles) la plus adaptée aux desseins des hommes, quel que soit le lieu et l’époque. Si cette grammaire était appliquée, elle le serait de manière imparfaite du fait que chaque homme est sujet à l'erreur, mais aussi à des volontés de transgression de par sa nature pécheresse.

Cette grammaire constitue alors les lois d'un système religieux, bien que parfois, comme pour le christianisme ou dans l'athéisme occidental, les lois civiques ne soient pas considérées comme des questions religieuses. Pour autant, dans les deux cas, ces lois civiques sont alignées à une éthique, donc il y a encore religiosité.

En revanche, dans le cas où cette grammaire est inapte à la réalisation des desseins humains ou finit par le devenir, la civilisation ou le groupe est anéanti, et l’on peut citer le passage de la Torah sur le veau d’or lors de l'exode du peuple juif.


\begin{center}
\begin{quote}
\textit{Yahvé dit alors à Moïse: "Va, redescends! Car ton peuple s'est corrompu, ce peuple que tu as fait sortir du pays d'\'Egypte! Ils n'ont pas été longs à se détourner de la voie que je leur avais enseignée: ils se sont fait un veau de métal fondu, ils se sont prosternés devant lui et lui ont présenté des sacrifices. Ils ont même dit: Israël, voici tes dieux qui t'ont fait sortir du pays d'\'Egypte!" \\ Yahvé dit encore à Moïse: "j'ai bien compris que ce peuple a la nuque raide. Laisse moi donc déchaîner contre eux ma colère: je vais les faire disparaître, mais je ferai sortir de toi une grande nation."}
\end{quote}
Exode, chapitre 32 verset 7 à 10.
\end{center}

Bien sûr, cet extrait de la Torah reste, selon moi, à interpréter d'un point de vue allégorique, comme une mise en garde contre l'égarement religieux, l'adoption de fausses croyances ou de règles néfastes qui mènent à leur perte ceux qui y adhèrent. Leurs actions individuelles et collectives deviennent alors inadaptées à la réalisation de leurs desseins, voire à leur survie. Je trouve ce passage riche d'enseignements.


\chapter{Opinion sur les prophètes et les écrits religieux}

De mon point de vue, les prophètes existent sans exister et il n'y a en réalité que des fous qui ont été écoutés. Pour autant, mon doute sur l'existence réelle des prophètes subsiste, car il reste délicat de rendre rationnels les miracles qu'ils ont accomplis, au-delà des propos qu'ils ont tenus.

Néanmoins, certains auraient pu être atteints de folie géniale, et leurs actions individuelles ayant mené à des actions collectives auraient tout autant été marquées par cette folie.

Les Grecs parlaient d'ailleurs de folie divine, là où, à notre époque, tout type de folie est perçu comme infructueux et reste systématiquement condamnable.

Mais il est tout à fait possible que certains de ces "fous" détiennent ou ont détenu des connaissances majeures sur le divin.

En assemblant les connaissances de ces individus, il est alors possible de construire un recueil composé de l'intégralité des connaissances humaines sur Dieu.

Sans doute ce recueil resterait-il très incomplet. D'ailleurs, avons-nous au moins un livre religieux réellement satisfaisant ?

Il reste encore que cette entreprise de construction de canon religieux est périlleuse, incertaine, sujette à l'erreur (n'en déplaise aux musulmans), divise et va parfois jusqu'à détruire des peuples.

\chapter{Solidité des dix commandements face à ceux de différents cultes}

Je pense que les dix commandements sont à la fois trop élémentaires et trop pertinents pour qu’il n’y ait pas au moins une part de divin en eux. Ces commandements constituent aussi, selon moi, le noyau des civilisations les plus saines.

Dans son ouvrage "Le Génie du Christianisme", Chateaubriand compare les différents équivalents aux dix commandements appliqués dans d’autres civilisations. Citons-en quelques-uns:

\vfill
\newpage
\textbf{Lois des Gaules ou des Druides.}

\begin{quote}
\begin{itemize}

\item "l’univers est éternel, l’âme immortelle.
\item Honore la nature.
\item Défendez votre mère, votre patrie, la terre.
\item Admets la femme dans tes conseils.
\item Honore l’étranger, et mets à part sa portion dans ta récolte.
\item Que l’infâme soit enseveli sous la boue.
\item N’élève point de temple, et ne confie l’histoire du passé qu’à ta mémoire.
\item Homme, tu es libre : sois sans propriété.
\item Honore le vieillard, et que le jeune homme ne puisse déposer contre lui.
\item Le brave sera récompensé après la mort et le lâche puni."

\end{itemize}
\end{quote}

\vfill
\newpage
\textbf{Lois Égyptiennes.}

\begin{quote}
\begin{itemize}
\item "Cnef, dieu universel, ténèbres inconnues, obscurité impénétrable.
\item Osiris est le dieu bon ; Typhon le dieu méchant.
\item Honore tes parents.
\item Suis la profession de ton père.
\item Sois vertueux ; les juges du lac prononceront après ta mort
sur tes œuvres.
\item Lave ton corps deux fois par jour, et deux fois la nuit.
\item Vis de peu.
\item Ne révèle point les mystères"
\end{itemize}
\end{quote}

\vfill
\newpage
\textbf{Lois de Pythagore.}

\begin{quote}
\begin{itemize}
\item "Honore les dieux immortels, tels qu’ils sont établis par la loi.
\item Honore tes parents.
\item Fais ce qui n’affligera pas ta mémoire.
\item N’admets point le sommeil dans tes yeux, avant d’avoir examiné trois fois dans ton âme les œuvres de ta journée. Demande-toi : Où ai-je été ? Qu’ai-je fait ? Qu’aurais-je du faire ?
\item Ainsi, après une vie sainte, lorsque ton corps retournera aux éléments, tu deviendras immortel et incorruptible : tu ne pourras plus mourir."
\end{itemize}
\end{quote}

\vfill
\newpage
\textbf{Les dix commandements (traduction de Louis Segond).}
\begin{quote}
\begin{itemize}
\item "Tu n’aura point d’autres dieux devant ma face.
\item Tu ne feras point d’image taillée, ni de représentation quelconque des choses qui sont en haut des cieux, qui sont en bas de la terre, et qui sont dans les eaux plus bas que la terre. Tu ne te prosternera point devant elles, et tu ne les serviras point; car moi, L’Éternel, ton Dieu, je suis un Dieu jaloux, qui punis l’iniquité des pères jusqu’à la troisième et à la quatrième génération de ceux qui me haïssent, et qui fait miséricorde jusqu’à la millième génération à ceux qui
m’aiment et qui gardent mes commandements. 
\item Tu ne prendra point le nom de L’Éternel, ton Dieu, en vain; car L’Éternel ne laissera point impuni celui qui prendra son nom en vain.
\item souviens-toi du jour du repos pour le sanctifier. tu travailleras six jours, et tu feras tout ton ouvrage. Mais le septième jour est le jour de repos de L’Éternel, ton Dieu : tu ne feras aucun ouvrage, ni toi, ni ton fils, ni ta fille, ni ton serviteur, ni ta servante, ni ton bétail, ni l’étranger qui est dans tes portes. Car en six jours L’Éternel a fait les cieux, la terre, la mer et tout ce qui y est contenu, et il s’est reposé le septième jour : c’est pourquoi L’Éternel a béni le jour du repos, et l’a sanctifié.
\item Honore ton père et ta mère, afin que tes jours se prolongent dans le pays que L’Éternel, ton Dieu, te donne.
\item Tu ne tueras point.
\item tu ne commettras point d’adultère.
\item Tu ne déroberas point.
\item Tu ne porteras point de faux témoignage contre ton prochain.
\item Tu ne convoiteras point la maison de ton prochain; tu ne convoiteras point la femme de ton prochain, ni son serviteur, ni sa servante, ni son bœuf, ni son âne, ni aucune chose qui appartienne à ton prochain."
\end{itemize}
\end{quote}

\begin{center}
***
\end{center}


Certaines religions n’admettent pas l’existence d’un libre arbitre et laissent penser que nous vivons dans l’illusion que nous exécutons nos choix alors que tout notre vécu est fixé d’avance par notre destinée. Mais je crois personnellement en l’existence du libre arbitre. Les Grecs et les Romains croyaient au destin, les Vikings tout autant, et les protestants calvinistes pensent que nous sommes des robots exécutant des actions tout à fait déterministes au regard de Dieu.

Pourtant, l’omniscience divine prête à penser que nos choix sont parfaitement connus au moins de notre Dieu; mais ce Dieu est aussi omnipotent, illimité en capacité créatrice, et peut donc très bien créer une pierre trop lourde à porter pour lui-même. Dieu peut donc tout simplement s’interdire de connaître une partie ou même l’ensemble des choix que nous effectuons.

Partant du principe que nous effectuons des choix individuels et collectifs, certains sont bénéfiques et d’autres mènent à des difficultés et des chaos dans les cas extrêmes. Si les choix bénéfiques sont des grâces divines et les choix néfastes des châtiments, je pense qu’il est judicieux de s’orienter vers des choix apportant des bénédictions, car d’une part, cela est plus confortable lors de notre vie terrestre, et d’autre part, potentiellement mieux rétribué une fois notre passage sur terre arrivé à son terme.

Il n’est cependant pas évident de mesurer les effets de ces choix, dont certains pouvant paraître bénéfiques dans l’immédiat se transformeront en calamité à moyen ou à long terme. Je fais donc le choix personnel de respecter au mieux les dix commandements, car on a la preuve par l’expérience (différentes civilisations ayant adopté d’autres modèles et servant de comparatif) que rien n’est plus profitable au développement humain dans le cas où ces commandements constituent le noyau éthique et comportemental d’une civilisation. L’application des dix commandements est donc nettement porteuse de bénédictions à court, long et moyen terme, même s’il peut être, dans certains cas, douloureux de s’imposer cette discipline. Les dix commandements assurent, à mon sens, plus de vitalité et de pérennité à n’importe quelle civilisation, quel que soit son climat et sa situation géographique, que n’importe quel autre équivalent. Il suffit d'observer le prestige civilisationnel des sociétés qui les ont appliqués (les Romains et les Grecs ayant été dépassés, leurs empires ayant fini par chuter).

Les lois de nos sociétés interdisent toujours le meurtre et le vol, c’est sûrement de bonne augure, mais qu’en est-il du reste ?

Que les hommes le veuillent ou non, tout individu ainsi que toutes ses créations matérielles et abstraites, et par extension les lois humaines, sont soumises aux lois divines et, comme je l’ai évoqué dans le chapitre 12, les déclins arrivent lorsque l’on s’éloigne trop dangereusement de ces lois inébranlables.

Pour ma part, ayant un point de vue proche de celui d’un chrétien à ce sujet, qu’il y ait ou non des lois poussant les individus à appliquer ces commandements m’importe peu. Je perçois ces commandements comme des lois morales que l’on décide ou non de s’imposer. Pour ce qui est du système juridique, citons simplement les propos du Christ : \begin{quote}\textit{"Rendez à César ce qui appartient à César et à Dieu ce qui appartient à Dieu."}\end{quote}

Pour digresser, cette parabole constitue une différence majeure qu'a le christianisme avec l'islam, qui impose une loi civique, tandis que le christianisme ne vient seulement que prescrire des lois morales et laisse le législatif à ceux qui détiennent le pouvoir. À mon sens, c'est une des raisons de la supériorité de cette religion en termes de prestige civilisationnel et de son aptitude à évoluer selon les époques qu'elle traverse depuis plus de deux mille ans.

Aucune preuve ontologique ne peut réellement appuyer ces propos. Je présume que le bon sens des dix commandements pourra être vérifié par chacun au fil des évolutions de nos sociétés et des enseignements tirés des cataclysmes futurs. Affaire à suivre donc...