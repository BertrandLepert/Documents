\part{Sur les principes de la création humaine et divine}

\chapter{Introduction : Des croyances religieuses et de ses dessins}

Certains savants définissent la religion par la croyance en une ou plusieurs divinités. Ainsi, les formes primitives de la religion, telles que l'animisme, ne sont pas considérées comme des religions. Pour parler de religion en des termes plus inclusifs, nous préférons la définir comme un ensemble de croyances considérées comme absolues, voire transcendantes. De cette manière, les religions primitives mais aussi l'athéisme occidental (possédant même cosmogonie, eschatologie et ontologie) peuvent être considérées comme des religions à l'instar des systèmes de croyances déistes. Il est à noter que tous ces systèmes de croyances sont vivement débattus et contestés, tout n'étant qu'une question de point de vue et de sensibilité. De plus, aucune religion encore pratiquée n'est figée, mais la plus sujette à des changements rapides reste l’athéisme occidental, car très appuyé sur des disciplines savantes modernes qui connaissent des évolutions rapides.

Ceci étant, tout individu, croyant ou non en l’existence du divin, possède donc un système de croyances. Ces croyances, exactes ou erronées, lui servent à rationaliser et orienter ses choix, à accomplir ses objectifs, et, au final, si ses actions sont adaptées, à atteindre les objets de ses désirs.

Ainsi, cet ensemble de croyances peut donc tout à fait être assimilé à une religion. La nature religieuse de l’être humain est donc indéniable et il lui est impossible d’y faire abstraction. En d’autres termes, chez l’homme, s'abstenir de toute religion lui est inaccessible.

Les individus se considérant comme non religieux sont donc dans la simple méconnaissance de leur caractère religieux, bien qu’ils n’ignorent en aucun cas posséder des croyances. Les croyances des plus hauts degrés (axiomatiques) servent à l'homme comme fondations lui permettant des raisonnements logiques et ainsi d’accéder à des croyances d’un ordre moins élevé. Ce sont les systèmes de croyances les plus exacts qui, employés à des actions concrètes, mènent aux desseins les plus adaptés. Mais si les croyances axiomatiques sont erronées, les raisonnements qui en découlent le sont aussi, et les desseins humains sont voués à l'échec.

Le succès des actions humaines, parfois grandioses, qu’elles soient individuelles ou collectives, reste, il me semble, le but premier de toute religion. Pour ma part, la faille de l'athéisme occidental réside dans le fait que ce système de croyances ne préconise aucune mœurs ni éthique autre que celles de son passé chrétien, sur lequel il s'est bâti et qui est de plus en plus oublié et attaqué.

\begin{center}
***
\end{center}

Si des croyances fausses sont enseignées délibérément à un ou plusieurs individus, c’est pour les contraindre à exécuter des actions que le menteur pense être la solution pour l'obtention de ses désirs. Le menteur peut également croire en des notions erronées qui lui ont été inculquées par un autre menteur ou un ignorant.

Manipuler un individu ou une population consiste donc à lui inculquer de fausses notions afin de lui faire exécuter les actions voulues par le manipulateur (si l’individu est convaincu qu’il y a un lion dehors, il ne sortira pas de sa maison). Bien sûr, plus le manipulateur est habile, plus le manipulé devient sa marionnette. Cela vaut autant pour un individu isolé que pour un phénomène de groupe plus ou moins conséquent.

Il y a manipulation lorsqu'un ou plusieurs individus inculquent à un ou plusieurs autres individus des notions soit habilement lacunaires (mensonge par omission), soit fausses afin de faire exécuter par le parti manipulé des actions qui arrangent le parti manipulateur.

Le double tranchant de la manipulation réside dans le fait qu'il reste possible que l'individu manipulé cherche à assimiler d'autres individus, que le manipulateur aurait voulu épargner.

Dans ce cas de figure, lorsqu'il s'agit de systèmes de croyances complets, il n'est plus question de religions, mais de sectes.


\begin{center}
\begin{quote}
\textit{"laissons au peuple le soin de croire que la science va réellement au fond choses."} \end{quote} Friedrich Nietzsche.
\end{center}

Ayant bénéficié d'un cursus universitaire scientifique, qui m'a conduit à travailler dans la physique et l'informatique, j'ai été surpris par le nombre d'ingénieurs et de chercheurs rencontrés qui vivent dans cette fausse croyance.

\chapter{Des principes véritables aux raisonnements qui découlent de ces lois axiomatiques}

La physique s’attache à déterminer des lois parfaites de phénomènes quantifiables et matériels qui s'observent dans la nature. Par conséquent, ces lois prennent obligatoirement racine dans des constats empiriques, car, en premier lieu, seuls nos sens et le jugement porté sur nos sensations nous permettent d’interpréter les phénomènes naturels. Sans sensations et sans capacité à les juger, nous n’aurions d’ailleurs aucune conscience de ces phénomènes.

Concernant les phénomènes microscopiques, des lois ont pu être déterminées, soit parce que ces phénomènes se manifestent à l’échelle macroscopique, soit grâce à la création d’instruments permettant d’observer puis d’interpréter ce qui se passe à cette échelle (comme, par exemple, avec l'accélérateur de particules).

Plus une loi découverte est fondamentale, plus elle est générale et englobe des lois caractérisant des phénomènes sous-jacents. Les lois les plus fondamentales ont donc une nature principielle, ce qui implique qu’elles sont impossibles à démontrer et sont donc utilisées comme axiomes. Les lois fondamentales sont donc toujours au plus proche des constats empiriques ou métaphysiques et servent de socle à des développements théoriques plus ciblés.

Par exemple, les trois lois de Newton, qui sont axiomatiques, sont utilisées comme principes fondamentaux lorsqu’il s'agit de phénomènes mécaniques et partent du constat que les forces exercées sur des objets matériels déterminent leurs mouvements. L’existence même du mouvement et des forces est constatée empiriquement, avec l'argument qu’aucun individu sain d’esprit ne conteste leur existence.

Le principe variationnel de moindre action, applicable à la mécanique et utilisé dans de nombreux domaines de la physique tels que l’optique, la thermique ou l’électromagnétisme, permet également de déterminer une trajectoire lorsque l’action employée à la produire est une donnée connue, et inversement. Ce principe est une définition contenant deux phrases littérales et est admis comme un phénomène naturel général. À partir de cette définition, il est alors possible de développer les calculs nécessaires à la quantification du phénomène.

Le principe de moindre action a été défini par Pierre Louis Moreau de Maupertuis au XVIIIe siècle:

\begin{center}
\begin{quote}
"L’Action est proportionnelle au produit de la masse par la vitesse et par l’espace. Maintenant, voici ce principe, si sage, si digne de l’Être suprême : lorsqu’il arrive quelque changement dans la Nature, la quantité d’Action employée pour ce changement est toujours la plus petite qu’il soit possible."
\end{quote}

\end{center}

Cette loi est donc une considération métaphysique, en premier lieu indépendante de tout problème physique.

C’est à partir de ces grandes lois axiomatiques que tout problème physique est considéré et résolu par une formulation mathématique. Il est arrivé dans l’histoire de la physique que certains principes découverts de manière empirique aient été justifiés par des principes plus généraux et se soient donc vu attribuer une démonstration théorique.

Je crois, comme Einstein le pensait, en une loi physique unique justifiant par le calcul théorique toutes les autres lois et les grands principes dont elles découlent. En revanche, je pense qu’il est impératif que cette loi soit un axiome indémontrable sur lequel s'appuient tous les phénomènes physiques.

Toutes les disciplines scientifiques fondamentales de la physique ou des mathématiques admettent des axiomes considérés comme indémontrables, émanant de considérations métaphysiques, et la découverte d’une loi unique pourrait les justifier par le calcul.

On voit donc que la découverte de cette loi unique reviendrait à comprendre parfaitement l’interaction divine, ou du moins le principe créateur et ordonnateur de l’univers, dans le cas où celui-ci n’a ni volonté ni intelligence. Pour ma part, je pense que toute création nécessite une volonté de créer et une intelligence organisatrice permettant de concrétiser cette action. Je trouve qu’il est suffisamment évident qu’il n’y a aucune raison pour que des choses se créent sans raison(s), et que si ces choses ont une raison d’être, c’est avant tout parce qu’elles ont été raisonnées.

Dans son ouvrage "De gravitatione", son auteur, Isaac Newton explique: 

\begin{center}
\begin{quote}
"On ne trouve guère d’autre cause à l’athéisme que cette notion de corps en tant que doté d’une réalité en soi, complète, absolue et indépendante."
\end{quote}
\end{center}


\chapter{De l’impossibilité de l’engendrement de l’univers par l’action seule du hasard}

Le hasard est un concept abstrait qui est employé pour déterminer le comportement d’un phénomène dont l’aspect déterministe reste incompris. Les lois de probabilité sont donc utilisées pour mieux comprendre les effets liés à une cause en partie ignorée. Un chiffre obtenu par un jet de dés, tout comme une combinaison gagnante au loto, obéissent en réalité aux lois physiques de la mécanique. Dans ces deux cas de figure, les forces exercées sur les dés comme sur les boules, et la géométrie de ces objets, si elles étaient parfaitement connues, rendraient tout à fait déterministes le résultat de ces tirages.

Connaissant tous les éléments du problème, les lois de Newton renverraient le résultat des tirages de manière systématique. Dans le cas où l’univers n’obéirait qu’à des lois physiques, aucune action ne renverrait un effet dû au hasard. Si l’on peut penser que tout est déterministe, il en résulte que le hasard n’est pas un phénomène ayant une existence concrète (au moins en ce qui concerne les phénomènes matériels).

Pour ceux qui croient, comme moi, au libre arbitre, cela implique que l’univers n’est pas totalement déterministe, puisque nous exécutons des choix impossibles à prévoir par des lois physiques qui ne sont pas non plus dus au hasard. De mon point de vue, l’univers n’est donc pas totalement déterministe, mais le hasard n’a cependant pas d’existence concrète. Si nos actions ne sont pas physiquement déterministes, cela est dû à notre capacité à effectuer des choix conséquents du fait que nous sommes des êtres conscients.

Pour les personnes croyant que le hasard possède une existence concrète, il reste tout de même impossible que la cause première de la création de l’univers provienne d’une probabilité mathématique. En première objection, tout concept (abstrait ou concret) nécessite d’être créé. Dans ce cas, l’univers ne peut être engendré par le hasard, car il y a alors nécessité d’un concept plus général devant engendrer lui-même l’existence du hasard. Il me semble qu’on ne peut pas raisonnablement penser que le hasard émane de lui-même ou a été engendré par lui-même.

En seconde objection, il est évident que le hasard, s’il existe, est un phénomène agissant sur des éléments préalablement présents et influençant leur comportement. Le hasard est donc tout à fait stérile s’il n’a aucun objet sur lequel exercer son action. Celui-ci ne peut qu’agir sur des objets déjà préexistants, ce qui implique qu’il ne peut pas créer ex nihilo ! Le hasard ne peut donc être un principe créateur.

Le hasard reste alors inapte à engendrer quoi que ce soit. Si celui-ci existait, il ne pourrait qu'avoir une influence sur des phénomènes impliquant des étants déjà existants. Pour un scientifique, les lois de la physique (même quantique) sont tout à fait déterministes, le hasard n'existe pas dans l'univers et n'est qu'un outil mathématique permettant d'estimer au mieux le résultat d'un problème lorsque certaines données sont inconnues.

Cependant, la physique ne s'attache qu'à traiter de la matière et de son mouvement. Cette discipline ne permet donc qu'une considération purement matérialiste de la nature et ne peut pas tout englober dans le cas où l'on croit que l'univers n'est pas uniquement composé de matière.

Si la physique vient infirmer l'existence du hasard en ce qui concerne la matière, il reste indéterminé que celui-ci ait ou non une influence sur des étants immatériels.

Nous pouvons alors tout de même citer une objection d'Albert Einstein sur la question du hasard dans la physique quantique:

\begin{quote}
"Dieu ne joue pas avec les dés."
\end{quote}

En d'autres termes, même si je crois en l'existence de phénomènes immatériels, je ne crois pas que le hasard puisse les influencer de quelque manière que ce soit. Donc, de mon point de vue, le hasard n'existe pas.

\chapter{Démonstration philosophique de l’existence du principe créateur et mise en exergue de son caractère intemporel}

Commençons par définir le néant comme l'absence de tout étant. Le néant étant de toute évidence une chose en soi, sa propre absence est incluse dans sa définition même. C’est la raison de son inexistence concrète. Le néant étant un concept, un concept n'est pas une absence de tout, c'est déjà quelque chose. Dans cet univers hypothétique, le néant en tant que concept, est-il alors omniprésent ou absolument inexistant ? Il semble bien que de son omniprésence suivrait son absence dès lors qu'il y aurait engendrement. Si l’on part de la genèse de l’univers avec un commencement où rien n’existe, ce serait le néant lui-même qui aurait pour charge de générer l’existence de tout étant. Ce néant serait-il alors anéanti pour autant ? Il est vraisemblablement impossible d’effectuer l’exercice dans le sens inverse, puisque la présence d’une ou plusieurs choses en soi a pour effet de rendre caduque l’hypothèse d’un néant omniprésent. Ce néant hypothétique devrait alors être à l’origine de tous les univers (matériels et immatériels), cohabiter avec toutes les choses puisqu'elles en seraient toutes issues. Mais au sens strict, l’absence de toute chose inclurait l’absence d’un néant en tant que concept. On voit là que l'inexistant est impossible en tous lieux et à tous moments, et que pour qu'il y ait de l'existence, il n’est plus possible de parler de néant originel, mais d’un principe créateur omniprésent spatialement et temporellement, ayant d'ailleurs créé ces deux concepts, et dont tout émerge.


\begin{center}
***
\end{center}

Mais alors, chaque étant est-il issu de ce principe un par un, de manière séparée, ou bien leur ensemble complet en est-il issu ? Il semble que les deux phénomènes cohabitent. En tout cas, un seul objet ou phénomène peut être issu de ce principe, tout comme un ensemble d’objets et/ou phénomènes de toutes natures. Si toute chose est générée ex nihilo en une seule fois, on peut alors considérer que l’univers est un et indivisible. À l’inverse, si les étants sont extraits un à un de ce principe, on peut envisager un univers fragmenté et inclure la possibilité d’isoler certains objets et phénomènes. Dans le premier cas, l’univers serait continu, tandis que dans l’autre, il serait discret, même s'il contenait une infinité d’éléments. La question (qui remonte à l’antiquité grecque) de savoir si l’univers est continu ou discret pourrait se résumer à cette approche ambivalente des objets et phénomènes issus du principe créateur, soit un à un, soit dans leur ensemble. Cela ne serait donc qu’une question de point de vue, dépendant de la considération de ses observateurs.

\begin{center}
***
\end{center}

Discontinuité dans la genèse d’un univers ? Absence de tout à l’origine des temps ($t = 0$), puis présence d'étants dès lors que le temps dépasse son origine d'une valeur infinitésimale ?

Que nenni, les objets et phénomènes sont issus du principe créateur en raison de l’impossibilité de l’existence du néant, telle qu’elle a été définie précédemment. On peut alors conclure qu'au minimum, le principe créateur a toujours été et est. Sera-t-il ? Est-il immortel, incorruptible ? Oui, puisqu’un retour à l’absence de toute chose impliquerait un retour au néant qui, comme nous l’avons vu, ne peut exister. Ce néant impossible est donc bien concret puisqu’il définit au minimum l’existence d’un concept. Il faut alors en conclure que le principe créateur à l’origine de l’univers a été, est, et sera. Soit traduit en hébreu, Jéhovah, un des noms employés dans les textes bibliques pour désigner Dieu.


\begin{center}
***
\end{center}

Il est admis que la genèse d’une représentation d’un concept de l'univers possède une fin ainsi qu’un début : l’idée. Deux cas de figure peuvent être considérés : l’idée est juste ou erronée.

Si l’idée est juste, sa forme est donc conforme à la nature de l’univers. Le concept a été, est et sera (il y a transcendance), tandis que sa découverte par l'homme prend un début et une fin. Le début est l’établissement de l’Idée, et sa fin, son aboutissement.

Si l’idée est fausse, celle-ci conserve un début et une fin. Le concept qui en découle, en revanche, possède une nature contraire à celle de l’univers. Celui-ci n’a jamais été, n’est pas et ne sera potentiellement jamais. Si l’univers évolue, il est pourtant probable que cette idée devienne juste. Dans ce cas, le concept existe sur une durée donnée. Peut-on dire que cette idée a été, est et sera ?

Dans un concept sur l'univers, du moins, il est possible d’y ajouter l’idée d’évolution. Ainsi, des Idées peuvent être ou avoir été justes, ou le devenir ultérieurement. Cependant, ce ne sont pas des lois applicables à la totalité de l’univers mais seulement à des instants donnés. Ces lois sont bornées dans le temps. Il est aussi possible de déterminer des lois bornées dans l’espace, et à la fois dans le temps et l’espace.

Or une loi fondamentale, transcendante, régissant l'univers, même si celui-ci est évolutif, a été, est et sera puisqu’elle devra prendre en compte cette idée même d’évolution et ainsi régir des lois sous-jacentes qui peuvent être bornées dans le temps et l’espace.

Ainsi, les fondements d’un univers ont été, sont et seront à tout instant. Ceci invalide la possibilité évolutive des lois fondamentales d'un univers, et donc celle du principe créateur.

\begin{center}
***
\end{center}

Les lois de l'univers (fondamentales ou non) dans leur globalité restent inconnues de l'homme en l'état. Ce sont des univers conceptuels qui représentent des fragments de l'univers global (matériel et immatériel). Pourtant, l'univers entier interagit avec l'homme et vice versa. On peut, dans un cas plus général, dire qu’un fragment d'univers, s'il n'est pas isolé, agit sur l'univers dans sa totalité, qui lui-même agit sur ce fragment. Si un phénomène est mis en place dans l'univers matériel, deux cas peuvent donc être considérés:

\begin{itemize}
\item Le fragment concerné où se situe l’origine du phénomène interagit sur l’univers qui lui même interagit sur le fragment et ainsi de suite... Le phénomène est ininterrompu (exemple du larsen ou encore de la bombe atomique).
\item Le fragment interagit sur l’univers qui, quant à lui n’exerce aucune rétroaction sur le fragment concerné. Le phénomène est directement interrompu en dehors du fragment concerné.
\end{itemize}

Ces deux cas sont bien sûr idéalisés puisque même une explosion atomique finit par s’atténuer. Dans l'univers matériel, nous n'avons donc que des cas intermédiaires plus ou moins importants de ces deux phénomènes possibles. La raison réside dans la transduction énergétique. L’énergie se conserve mais se dissipe sous d’autres formes autres que celle du phénomène initial. Cette énergie est réutilisée pour donner lieu à d’autres phénomènes extérieurs au fragment d'univers considéré. Il est donc tout à fait possible d’émettre l’hypothèse d’une source unitaire et primordiale d’énergie se diffusant pour donner lieu à des phénomènes multiples, réalimentant eux-mêmes d’autres phénomènes dans une boucle infinie.

\begin{center}
***
\end{center}

Y a-t-il alors possibilité d’une source ponctuelle d’énergie à l’origine des temps ?

Nous avons démontré plus haut que l’univers n’a pas d’origine temporelle en raison de l'existence du principe créateur en tout temps. Cependant, il n’est pas impossible que le temps, élément de cet univers et induit par son principe créateur, ait une origine. Lorsqu’il ne s’agit pas des fondements de l’univers, l’évolution est possible, et le temps peut aussi passer de l’inexistant à l’existant (peut-être même à plusieurs reprises).
Soit deux possibilités:

\begin{itemize}
\item Le temps est un fondement de l’univers et n’a pas d’origine ( il a été, est et sera).
\item Le temps n’est pas un fondement de l’univers et a une origine, n’a pas toujours existé (contrairement à son principe créateur). 
\end{itemize}

Or le temps est mesurable et évolue, et seul son concept en tant qu’Idée est inchangeant. En tant que phénomène, le temps est sous-jacent au(x) fondement(s) de l’univers, puisque tout ce qui est évolutif fait partie des concepts qui ne sont pas fondamentaux à la genèse d’un univers.

L’énergie elle-même évolue, notamment en fonction du temps, se dissipant et générant d’autres phénomènes. La matière évolue par le mouvement, et la masse varie en raison des mouvements mêmes de la matière. Pourtant, les Idées de ces concepts sont invariantes.

On sait que matière, énergie, et masse sont trois concepts physiques qui se conservent et ne font qu’évoluer sur le plan matériel. Nous avons démontré que l’existence d’un néant impliquerait au minimum la correspondance à un concept possédant une existence concrète, ce qui est un non-sens. Cela ne peut donc qu’impliquer l’existence d’un étant invariant selon tous paramètres, et à l’origine de toutes choses (lois, phénomènes, matière, énergie, temps, etc.). N’étant pas soumis au temps, cet étant est incorruptible.

En résumé, tout évolue sauf le principe créateur : Dieu et ses Idées. Le mouvement, le temps, et la matière (intrinsèquement liés et évolutifs), restent toujours des constats empiriques utilisés comme axiomes par les physiciens, et sont donc indéniablement des axiomes démontrables si l'on possédait la connaissance de cette loi unique régissant l'univers évoquée dans le chapitre 1.

\chapter{De l'existence d'un Dieu monothéiste par l'extrapolation du principe de moindre action}

Commençons par rappeler le principe de moindre action tel qu'il a été défini par Pierre Louis Moreau de Maupertuis au XVIIIe siècle:

\begin{center}
\begin{quote}

« L'Action est proportionnelle au produit de la masse par la vitesse et par l'espace. Maintenant, voici ce principe, si sage, si digne de l'Être suprême : lorsqu'il arrive quelque changement dans la Nature, la quantité d'Action employée pour ce changement est toujours la plus petite qu'il soit possible. »

\end{quote}
\end{center}

Ce principe reste fondamental dans la physique et, dans le cadre strict de cette discipline, permet une approche énergétique des lois du mouvement des corps. Par vulgarisation, on peut dire que lorsqu'un mouvement opère, celui-ci emploie le minimum d'énergie possible à sa réalisation.

Puisqu'une action créatrice devrait nécessiter de l'énergie, sans doute elle-même créée en amont, il n'est pas déraisonnable d'émettre l'hypothèse de l'extrapolation de ce principe pour expliquer l'action créatrice de l'être suprême.

Créer un univers, même originel, puisque le hasard n'a pas d'existence concrète, n'est en rien dû à un "coup de chance" (cf. chapitre 3). Soit l'univers émane de lui-même, soit il a été créé, a évolué, et est, dans ce cas, soumis au changement. Or, s'il n'y a aucune raison à ce changement, c'est bien que l'univers change sans raison. S'il change sans raison, c'est qu'il n'a aucune raison de changer.

Pourtant, nous avons vu par le principe de moindre action que l'univers, lorsqu'une action est effectuée, tend à minimiser la dépense énergétique nécessaire. Aucune action superflue n'est alors effectuée.

Par conséquent, si l'univers n'avait aucune raison d'être en changement, ce changement serait superflu et n'aurait donc pas lieu. Cela montre que l'univers possède donc une raison à son changement. Il semblerait qu'une raison supérieure, exclue de l'univers, autorise ce changement ou mouvement, lui donne une raison, et, en ce qui concerne la matière, interdit toute autre action inutile ou superflue (pas de mouvement s'il n'y a pas de raison nécessaire).

Si le changement a donc une raison d'être, c'est bien qu'il a été raisonné par un Dieu, et l'hypothèse de plusieurs divinités serait d'ailleurs également superflue et donc invalide.

\chapter{De l’existence d’un repère spatial absolu de l’univers et des mouvements véritables}

Le mouvement des corps s’observe de façon relative par rapport à un lieu soit réellement fixe, soit considéré comme fixe. Cette notion est connue depuis l’antiquité, car Aristote, en traitant de la notion de lieu et de mouvement dans son ouvrage "Physique", démontre philosophiquement cette notion pouvant être considérée comme pré-relativiste.

Soit un train A à l’arrêt par rapport au repère terrestre et un train B suivant une trajectoire quelconque par rapport à ce même repère (ici considéré comme fixe). Dans ce cas, c’est le train B qui paraît en mouvement relativement au train A. De la même manière, si un observateur au repos dans le train B observe le train A et le décor, celui-ci verra ces éléments en mouvement, et s’il se considère comme fixe dans l’espace, il admettra que le train A et son décor sont animés d’une vitesse.

Dans l’histoire, les philosophes ont longtemps débattu de l’existence d’un repère spatial absolu. Si ce repère existe, cela implique qu’au moins un point de l’espace est réellement fixe de manière absolue et permettrait donc d’observer les mouvements tels qu’ils sont en réalité. Pour Aristote, un repère fixe existe bel et bien et est dû à l'implication d'un dieu qui crée le mouvement originel, moteur de tous les autres.

Descartes, cependant, ne concevait que des mouvements relatifs par rapport à des lieux et ne définissait d’ailleurs l’espace que par la présence d’objets plus ou moins éloignés les uns des autres. Newton défait par la suite cet élément de la doctrine cartésienne en apportant l’argument que faire tourner l’étendue du ciel étoilé autour de la Terre n’est pas le même phénomène que faire tourner la Terre sur elle-même et observer depuis celle-ci que le ciel est en rotation, car cela n’implique ni les mêmes grandeurs en termes de forces, ni les mêmes points d’application de ces forces.

Pour Newton, ce repère absolu existe donc bel et bien et peut être déterminé par la connaissance de toutes les forces appliquées à un objet de l’espace. Cette notion à priori juste renoue donc avec le point de vue aristotélicien d’un repère spatial absolu de l’univers, et nous pouvons alors observer que la Terre tourne réellement sur elle-même par rapport à ce repère absolu en mesurant la force de Coriolis.

Pour Newton, moi-même, et tout autre individu déiste et monothéiste, si une force semble provenir sans autre conséquence qu’elle n’existe que par elle-même, c’est que Dieu l'applique directement de par sa volonté, ou bien qu'une mise en mouvement primordiale est responsable de ces forces engendrant tous les autres. Bien qu'Einstein ait ajouté une conséquence nouvelle aux forces astronomiques par la déformation de l'espace-temps due à la masse des objets célestes, cela n’exclut pas une nature divine de ces phénomènes.

Mais les forces peuvent être la conséquence de phénomènes plus élevés se situant entre Dieu et elles-mêmes. Dans ce cas, l’action de Dieu la plus directe ne serait pas dans la création, le maintien, et le transfert des forces, mais dans quelque chose qui aurait notamment pour conséquence de créer des forces et du mouvement. On sait d’ailleurs que la présence de matière implique une attraction gravitationnelle, et, depuis Einstein, par la déformation du continuum espace-temps induite par sa masse. Donc, une mise en mouvement primordiale à l'origine des temps pourrait être due à la création originelle de matière.

Par analogie avec les êtres vivants, l’homme et l’animal peuvent mettre des objets en mouvement par la volonté. Mais cette volonté a d’abord pour effet d’actionner une mécanique corporelle analogue à celle d’un automate pour transmettre ses forces aux objets environnants. Une différence entre l’homme et les espèces animales est que l’homme possède une maîtrise supérieure de ce à quoi ces mouvements peuvent lui servir et détient donc une raison plus divine et raisonnable lui permettant d’exécuter ces actions.

Je me demande donc si Dieu maintient l'existence de forces et de mouvements par volonté directe ou si ses lois ont été établies préalablement et qu'aucune action de sa part n'est exécutée par la suite.

La mécanique classique, dans le cadre de cette discipline, a renvoyé Dieu à l'idée d'un horloger qui a établi ce système sans exercer d'interaction autre que sa création. Mais est-ce aussi la conséquence d'une volonté raisonnable à la présence de ces mouvements qui nous dépasse? Quel est le dessein de toute cette mise en mouvement?

Les êtres pensants, autres que Dieu, sont aussi les seuls à pouvoir influencer par la volonté, puis par l'action, ce mécanisme gigantesque qui, autrement, ne serait soumis qu'à la seule volonté divine. Dieu, ou le principe créateur, permet donc aux êtres vivants, et eux seuls, de mettre en mouvement ou de faire dévier, par l'exercice de leur libre arbitre, de manière consciente, des objets matériels dont les trajectoires seraient sinon uniquement dépendantes de(s) force(s) céleste(s) originelle(s).

\chapter{La trinité, l'un et le multiple}

En théorie des nombres, il faut d'abord considérer deux nombres si l'on veut engendrer tous les entiers par le calcul. Soit, par exemple, l'ensemble ${2,3}$ où le chiffre trois ne peut en aucun cas être engendré. Platon appelait ce chiffre le "chiffre sans père" car impossible à recréer sans le considérer directement. En théorie des nombres, pas de passage de l'un au multiple sans préalablement considérer le chiffre trois.

Que les nombres aient ou non une existence concrète reste, à mon sens, une grande question. Sont-ils une simple abstraction humaine ou existent-ils concrètement par acte de création extérieure en tant qu'immatériels?

Sans l’existence d'un démiurge, sans architecte de l'univers, les nombres ne peuvent exister qu'en tant qu'abstraction humaine dans le but de dénombrer et de donner du sens à des étants qui en seraient dépourvus. En revanche, si Dieu existe, et bien qu'il reste possible que celui-ci ait établi sa création autrement que par l'utilisation de nombres, en utilisant d'autres règles, il reste probable que les nombres aient une existence concrète en tant qu'immatériels et que, par exemple, l'espace tridimensionnel dans lequel nous sommes projetés possède son unité de mesure divine. Peu probable dans ce cas que le mètre ou le pied aient été employés.

Dans ce cas précis, pour des raisons de justesse conceptuelle, il serait commode de déterminer ces unités et règles de mesure et de les employer dans nos calculs scientifiques.

\begin{center}
\begin{quote}
\textit{"Dieu c'est la juste mesure des choses."}
\end{quote} Platon.
\end{center}

Autant donc s'employer à mesurer avec justesse si l'on veut mieux comprendre les phénomènes naturels. Pour ma part, je crois que l'architecte a créé et utilisé les nombres dans sa création. À ce jour, l'homme n'a pas trouvé de système plus approprié que le système métrique pour les calculs physiques; produit de la Révolution française, les savants qui se sont attelés à son élaboration considéraient qu'il serait employé par les scientifiques du monde entier en raison de sa supériorité. De nos jours, ce système reste indétrônable. Les sept unités physiques primordiales du système métrique, permettant le calcul de toutes les autres, sont : le mètre, le kilogramme, la seconde, l'ampère, le kelvin, la mole et la candela. J'ai personnellement l’impression qu'il devrait être possible de quantifier tous les phénomènes : courants électriques, température, quantité de matière et lumière, uniquement avec des unités de distance, de poids et de temps. Soit trois unités au total.


\begin{center}
***
\end{center}

Lorsqu'il s'agit de concepts matériels, observables dans la nature, tels qu'une couleur unique, celle-ci peut toujours se décomposer en trois couleurs primaires, et nous avons alors de nouveau un exemple du passage de l'un au multiple par la présence du chiffre trois. L'ensemble infini des couleurs peut être recréé à partir de trois autres couleurs primitives.

Ce concept métaphysique se retrouve aussi dans le principe trinitaire de la religion chrétienne et vient justifier le passage de l'un au multiple dans la création divine unique.

Au moins une autre mythologie intégrant ce savoir est la religion scandinave où le symbole du Valknut, représentant trois triangles entrelacés, et donc trois figures à trois côtés, donne une indication sur un degré estimable en termes de connaissance métaphysique qu'a pu établir ce peuple, puisque apparemment, un principe créateur basé sur des multiples de trois était considéré.


\begin{center}
***
\end{center}

Le concept de dyade inventé par les philosophes grecs met en avant le fait que certains concepts subjectifs existent de manière intrinsèquement liée à leurs contraires. L'un ne peut exister sans l'autre, comme le lourd et le léger, le dur et le mou, ou encore le clair et le sombre. Mais cela ne s'applique qu'à des considérations purement subjectives, puisque le lourd et le léger n'existent que relativement à la valeur de la masse de l'objet, le dur et le mou par la raideur, et le clair et le sombre par la quantité de photons réfléchie par la surface observée.

Enfin, la monade, conceptualisée par Pythagore, renvoie à l'unicité cohérente de l'univers incluant toute chose. Cette idée peut aussi bien renvoyer à Dieu qu'à un univers englobant tout et émanant de lui-même, mais aussi à la notion de panthéisme, où tout, y compris la matière, fait partie de Dieu.

Alors quoi ? Si, à ce moment-là, tout est interprétable comme un langage, voire simplement une harmonie numérique, l'univers aurait-il été composé sur une rythmique ternaire ?

Il semblerait que chaque phénomène naturel possède cet aspect ternaire ou trinitaire. En mécanique, force, masse et mouvement sont liés. En électromagnétisme, il en est de même pour le courant, le champ magnétique, et la force induite. On peut encore citer la célèbre formule fondamentale d’Einstein $E = mc^2$. Toute loi fondamentale de la physique ne fait intervenir que trois termes.

Après tout, la Bible enseigne que Dieu est une parole et que toute parole possède sa métrique.