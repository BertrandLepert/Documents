\part{Sur la négligence et le dévouement à Dieu}

\chapter{La grande usurpation savante}

Il serait bon de stopper cette trop forte distinction entre sciences dites "dures" et sciences dites "molles", surtout en ce qui concerne la philosophie. D'ailleurs, des philosophes tels qu'Emmanuel Kant écrivaient la philosophie comme des mathématiques pures. Il serait judicieux de rappeler les bouleversements scientifiques et paradigmatiques de la philosophie comme celles des religions.

Le christianisme a, par exemple, éliminé les croyances aux naïades, aux nymphes, etc., et très tôt dans sa longue histoire, à la plupart sinon toutes les créatures mystiques inexistantes. De ce fait, grâce à cet apport religieux, la nature a grandement perdu de son mystère et par conséquent, une partie importante de son mysticisme.

Toutes ces évolutions, scientifiques, philosophiques et théologiques ont globalement cheminé vers plus de justesse (comme évoqué dans le chapitre 8). Malheureusement, à l'époque actuelle, nos élites savantes semblent majoritairement enseigner que religion et philosophie sont au mieux surannées, sinon mortes.


\begin{center}
\begin{quote}
\textit{"Dieu est mort."}\end{quote} Friedrich Nietzsche.
\end{center}



\begin{center}
\begin{quote}
\textit{"La philosophie est morte."}\end{quote} Stephen Hawking.

\end{center}


\'Eliminer la spiritualité par et pour le spirituel afin d'atteindre plus de profondeur métaphysique ?!

Puisqu'on cherche à enseigner que la philosophie est morte ou, au mieux, tombée en désuétude, alors que cette assertion résulte d'une ébauche maladroite de raisonnement philosophique (puisque non prouvé mathématiquement), j'attends de la part des élites qui diffusent cette idée la démonstration mathématique de la soi-disant mort de la philosophie, en leur souhaitant un prompt succès !

Bien sûr, et heureusement, cette idée n'est pas partagée par l'ensemble des doctes contemporains.

La physique, le calcul mathématique, formel ou numérique, sont des sciences qui ne servent pas à elles-mêmes et, nous l'avons vu, n'ont pas d'autres applications concrètes que la construction de machines, de ponts, de routes etc... (cf. chapitre 8).

Ces assertions contemporaines n'ont que pour effet de discréditer et amoindrir la portée des propos des philosophes et des théologiens qui, par leur volonté de compréhension du monde, sont et resteront, pour moi, les artisans de premier ordre des avancées humaines des degrés les plus élevés. Avancées qui ne doivent d'ailleurs pas rester purement techniques.

La philosophie est-elle morte ? Laissons alors entre les mains de nos mécaniciens le soin de se charger, grâce à leurs connaissances techniques, des questions éthiques, politiques, métaphysiques... Je me permets aussi de leur rappeler que tous les principes de la physique classique encore en vigueur sont eux-mêmes enracinés dans la philosophie naturelle, elle-même enracinée dans l'ontologie biblique, et n'ont pas de démonstration autre (cf. chapitre 2).

Laissons-les donc enliser nos civilisations dans un désastre spirituel total afin de permettre à tout un chacun de constater, une bonne fois pour toutes, le péril de la démarche par le déclin.

Demandons aussi aux historiens compétents si l'expérience n'a pas déjà été tentée. Ce n'est certainement pas la première fois que l'homme essaye en vain de défier ou d'abandonner Dieu, avec les conséquences funestes chroniquées dans les écrits religieux et notamment bibliques. N'est-il pas stupide de répéter sans arrêt les mêmes erreurs ?


\begin{center}
***
\end{center}

Selon Aristote, la hiérarchie des sciences théoriques est établie selon leur degré de généralité et d'abstraction, et elles sont classées de la manière suivante :

\begin{itemize}
\item La métaphysique : de plus haut degré, étudiant l'être en tant qu'être, elle explore les concepts les plus généraux et abstraits.
\item La physique : étudie la nature et les principes fondamentaux qui gouvernent la matière.
\item Les mathématiques : de dernier degré, traitant des quantités dénombrables et des relations abstraites.
\end{itemize}

Ainsi, la physique et les mathématiques se trouvent à des degrés inférieurs à la métaphysique, qui est avant tout l'apanage des philosophes et des théologiens. Physique et mathématiques sont contraints de prendre racine dans la métaphysique.

\begin{center}
***
\end{center}

Que nos spécialistes des mouvements de la matière m'expliquent enfin la cause de ces mouvements par la preuve mathématique ou physique, alors que pour l'instant, le mouvement reste encore un constat empirique. Qu'ils répondent enfin à la question et apportent la preuve que tous les peuples se sont religieusement posée : avons-nous réellement un libre arbitre ou effectuons-nous des actions totalement déterministes ? L'univers est-il borné ou infini et, s'il est borné, qu'y a-t-il de l'autre côté de ses limites, puisque nous avons vu que le néant est un concept qui ne peut exister, puisqu'un concept n'est pas du néant ? Le monde que l'on observe existe-t-il concrètement ou est-ce une illusion ?

Ces questions n'ont jusqu'à présent été traitées de manières diverses au cours des siècles que par des philosophes et des prêtres. Les réponses apportées sont encore admises par les physiciens et mathématiciens dans la tradition de notre pensée religieuse helléno-chrétienne. Je suis sidéré de voir à quel point la plupart des scientifiques contemporains l'ignorent et admettent ces dogmes comme s'ils étaient des choses innées ! Or pas de réponse philosophique certaine, et encore moins physique, n'est réellement établie. Seule reste la croyance religieuse pour y répondre, ce qui implique aussi la possibilité d'erreurs.

\begin{center}
***
\end{center} 

Je ne prétends heureusement pas connaître le sens de la vie, mais je ne peux pas croire qu'elle n'en possède aucun. Si la vie n'avait aucun sens, elle n'aurait aucune raison d'exister (extrapolation du principe de moindre action du chapitre 5).

Dans nos sociétés contemporaines, avec l'émergence de l'humanisme et de l'athéisme occidental, le physicien et le mathématicien ont pris la place des doctes chargés de nous enseigner une ontologie, une métaphysique, et même une religion qui n'offre aucun sens à l'existence. Ces disciplines scientifiques doivent pourtant leur établissement et leurs lois fondamentales à une ontologie, et l'ontologie doit ses fondements à une religion.

Erreurs après erreurs, victoires après victoires, les religions et leurs ontologies ont évolué et ont aussi été bouleversées ; elles ont certainement gagné en vérité, mais restent imparfaites car créées par l'homme. Les évolutions et victoires de la métaphysique, de la physique et des mathématiques en sont une conséquence.

Vous l'aurez compris, pour ma part, les "sciences dures" ont usurpé de manière illégitime et dangereuse des rôles qui ne les concernent en rien et cherchent à remplacer une discipline dont elles dépendent et à laquelle elles doivent leur existence. 

\begin{center}
\begin{quote}
\textit{"Un peu de science éloigne de Dieu, beaucoup de science y ramène."}\end{quote} Francis Bacon.
\end{center}

\chapter{Sur l’application de la théorie du surhomme dans les sociétés contemporaines}

Suite à l’engouement porté à la théorie évolutionniste de Darwin ainsi qu’aux valeurs athées apportées dans toute l’Europe par la Révolution française, Nietzsche propose alors un dogme destiné à remplacer la religion chrétienne. Considéré par certains comme un prophète à l'origine de l'ère industrielle, ce philosophe incite ses lecteurs à adopter un système de valeurs basé sur l’athéisme, où l’individu et ses propres désirs deviennent alors son unique référentiel.

Influençant grandement le régime nazi, qui, contrairement à nos sociétés modernes, se base sur une interprétation racialiste et antisémite de ces écrits philosophiques, Nietzsche influencera aussi le mouvement humaniste animant aujourd’hui, en partie, les valeurs éthiques et morales des sociétés occidentales contemporaines.

Il est à noter que Nietzsche possédait d'abord une formation de philologue et donc une connaissance très élaborée des textes anciens, notamment religieux, dont sa philosophie est inspirée.

Le système de mœurs qui, selon lui, serait le plus judicieux à adopter est principalement inspiré de celui employé par les anciens peuples scandinaves de l’Europe du Nord. Par comparaison avec les sociétés occidentales actuelles, on peut mettre en évidence certains éléments tels que le débridement sexuel, ses valeurs féministes, sa tendance à l’individualisme, et aussi une certaine orientation idéologique prônant la "mort aux faibles". Il est tout de même important de souligner que les anciens peuples d’Europe du Nord prenaient soin des malades et de leurs infirmes en leur assurant une vie décente.

Pourtant, les valeurs morales et les mœurs de nos sociétés semblent souvent perçues comme bienfaisantes par une majorité et proviendraient d’une vision libérale nous menant vers un monde meilleur. Sans aller jusqu’à exiger que cette majorité lise des écrits philosophiques ni même que ces personnes comprennent Nietzsche (qui est tout de même considéré comme difficile, voire impossible à saisir),

Ces écrits tiennent plus de dogmes destinés aux élites et appliqués aux peuples occidentaux de manière voilée.

Je pense qu'il serait tout de même urgent pour ces personnes de considérer que l’on ne sort ni mœurs ni éthique d’un simple chapeau ! Si l’être humain commettait un jour la stupidité de faire le bien en ne se référant qu’à ce que ses pulsions lui dictent, nous signerions certainement l’éradication de notre espèce... Cela peut notamment s'observer par les dérives actuelles du mouvement "woke", qui, se voulant bien-pensant, mène aussi à des formes extrémistes dangereuses.

\begin{center}
\begin{quote}
\textit{"L'enfer est pavé de bonnes intentions."}\end{quote} Bernard de Clairvaux.
\end{center}

Le dogme nietzschéen est avant tout une amplification de celui apporté par la mythologie nordique et provient aussi d'une étude approfondie et philologique des phénomènes sociaux liés au déclin civilisationnel de différents peuples. Il est à noter que le paganisme des pays nordiques était à l’époque perçu comme satanique par l’Église catholique, puisque totalement contraire aux standards des valeurs chrétiennes les plus fondamentales.

Il est aussi possible d'effectuer un parallèle avec les évolutions rapides de nos sociétés modernes et celles des Vikings, qui ajoutaient tous les siècles environ, une nouvelle divinité à leur panthéon, et donc de nouvelles règles à suivre.

J’ai la forte appréhension que l’application de la théorie du surhomme générerait un déclin sans précédent si nos dirigeants devenaient réellement immodérés dans l’application de ces écrits. Fort heureusement, son œuvre est interprétée de façons très diverses et, je l'espère, rarement de manière aussi alarmante.

Pour ma part, je ne vois pas Nietzsche comme un véritable athée (ce serait mal le comprendre) mais plutôt comme un agnostique fou furieux dont l’œuvre, si son application devenait drastique, permettrait de vérifier par l'expérience sociale, l’existence du Dieu de l’Ancien et du Nouveau Testament, mais aussi de la validité de la théorie évolutionniste.

L’aboutissement de la théorie du surhomme, telle qu’elle est proposée dans l’ouvrage "L'Antéchrist", serait l’adoption de mœurs purement satanistes basées sur une inversion totale de celles du christianisme. L’homme s’infligerait alors par lui-même les mœurs les plus inadaptées à sa survie que l’on puisse imaginer, et son environnement deviendrait alors le plus rigoureux et le plus meurtrier possible.

La conclusion s’observe donc directement dans le titre de son ouvrage. Appliquer totalement la théorie du surhomme aurait alors deux effets possibles : soit, dans le cas où le Dieu biblique existe, nous pourrions observer l’arrivée de l’Antéchrist et la réalisation des prophéties eschatologiques de l’Ancien et du Nouveau Testament ; soit, dans le cas où ce Dieu n’existe pas et que la théorie évolutionniste est valide, accéder à une post-humanité plus puissante par la sélection naturelle, et donc au fameux surhomme de la théorie sensée l'engendrer.

\begin{center}
\begin{quote}
\textit{"Qu’est le singe pour l’homme ? Une dérision ou une honte douloureuse. Et c’est ce que doit être l’homme pour le surhomme : une dérision ou une honte douloureuse."}\end{quote} Friedrich Nietzsche, \\ ainsi parlait Zarathoustra.
\end{center}

Par cette interprétation, il est un peu mieux possible de constater l’immense génie pervers de Nietzsche, potentiellement cause de la démence qu’il a contractée. Par son apport, nous vivons désormais comme les cobayes d'une expérimentation funeste. Fort heureusement, toutes nos élites ne semblent pas s'attacher à ses considérations. Affaire à suivre...


\chapter{Négation de l’idée la valeur humaine}

Comme pour tout cet ouvrage, cette considération n’engage que moi et ceux qui pensent de cette manière. J’estime d’ailleurs avoir développé cette idée trop tardivement au cours de ma vie.

Je rejoins tout à fait l’opinion de Montaigne selon laquelle il n’y a pas de grands hommes mais simplement des hommes. L’idée que l’on peut se faire de la grandeur humaine se construit à partir de qualités désirables qui dépendent, dans une certaine mesure, de la considération de chaque individu. Cette idée de grandeur d’âme peut alors se développer par rapport à certains critères tels que la vertu, la quantité d’argent, le pouvoir, les capacités intellectuelles ou la force physique. Certaines de ces qualités peuvent être perçues comme perverses ou non en fonction des individus, de leurs croyances et de leurs cultures.

À mon goût, il y aura toujours trop de personnes s’adonnant à ces jugements de valeurs. Mais pourquoi émettre moi-même un jugement sur ces personnes ? Je souhaite quand même rester au mieux conforme à cette idée.

De mon point de vue, tout individu possède un parcours de vie digne d’intérêt qui lui est propre. S'il y a un sens à la vie de manière absolue, il est connu de Dieu seul. Dans ce cas, seul le fait de trouver et d’atteindre les objets de ses désirs pourrait compter, et la nature humaine est telle que nous resterons tous éternellement insatisfaits.
\begin{center}
\begin{quote}
\textit{"Il n'y a de plaisir que dans le boire et le manger."}
\end{quote} Livre de l'\'Ecclésiaste
\end{center}

Ces intérêts peuvent totalement diverger d’une quelconque volonté de puissance. Tout le monde s’est déjà brûlé en persistant dans cette voie. Comment pouvons-nous savoir objectivement ce qui est louable en traitant des données tout à fait subjectives ?
Que ces individus aient la santé, des richesses, de l'intelligence ou de la gloire
n’y change rien.

Je trouve que tout parcours de vie a une valeur initiatique et qu'il n'y aurait pas de réelles inégalités de destins, seulement des parcours de vie différents de valeurs équivalentes. Je vois la magnanimité, la puissance et la grandeur comme des illusions collectives servant d'œillères à la vanité de notre existence.

Pour ma part, je m’intéresse très peu à l’argent et au pouvoir. Je désire surtout une santé que je ne peux malheureusement pas atteindre pleinement et mon élévation spirituelle. Le simple fait de me procurer des livres ou de m'adonner à des activités intellectuelles est un vrai levier d’action me permettant d’évoluer dans cette voie.

Ces désirs n’engagent que moi et ceux qui pourraient en être intéressés. Avancer, même assez lentement, sur ce chemin n’est ni plus ni moins digne que d’avancer vers autre chose. À chacun sa vanité, et voici la mienne.

J’ai tout de même en dégoût les individus pervers prenant plaisir au mal et se glorifiant de pouvoir faire souffrir. L’essentiel pour moi reste alors d’éviter au mieux ces comportements et d’essayer de me corriger lorsqu’il m’arrive d’emprunter ce sentier.


\chapter{Conclusion: mon testament, pacte avec Dieu}

Peu importe la cause de mon décès, il reste néanmoins possible qu’il soit dû au déclenchement d’une guerre. Dans ce cas, je donnerai volontiers ma vie pour préserver certaines de mes vertus et éviter mon avilissement.

Quoi qu’il puisse arriver, quel que soit le contexte dans lequel je serai projeté, je respecterai toujours au mieux les dix commandements (cf. traduction de Louis Segond au chapitre 13).

De par ma nature pacifique, les commandements que je suis sûr de respecter au mieux sont ceux qui prescrivent de ne pas tuer et de ne pas voler.

Il m’est arrivé de mentir, de commettre des adultères, d’en vouloir à Dieu, et je n’ai, de par mon éducation athée, pas toujours cru qu’il y avait un Dieu.

Mais je fais le serment de ne pas tuer, même en cas de légitime défense. En cas de guerre ou de simple confrontation avec une autre personne ou un groupe, je pourrais me défendre, mais je refuserai toujours de prendre la moindre vie. Il se peut que je cherche à fuir pour prolonger ma vie terrestre. J’ai d’autres projets que celui de mourir de manière stupide. Il est cependant hors de question de m’adonner à des actions trop basses pour prolonger ma vie. Tuer ou voler sont donc pour moi totalement exclus, même en cas d’absolue nécessité.

Pour le reste, je ferai de mon mieux pour satisfaire ces commandements. Je ne suis pas convaincu du caractère divin du Christ, mais j’ai tout de même la conviction d’un Dieu omnipotent et de l’immortalité de l’âme. Lorsque celle-ci est liée au corps et au monde terrestre, elle laisse une empreinte indélébile lors du passage à l’étape suivante qui suit la mort du corps (cf. chapitre 10).

Je n’ai pas une grande affinité avec le bouddhisme et l’hindouisme, et mes connaissances concernant ces religions sont très limitées (je ne dispose malheureusement pas d’un temps infini pour les étudier). Je crois que ces religions sont tout à fait louables et pacifiques, mais je ne crois pas en la réincarnation.

Le peuple grec, que j’ai pu étudier à travers certains de leurs écrits philosophiques, y croyait. Socrate a cependant eu, par la suite, une influence considérable vers l’idée d’un Dieu unique, omnipotent et démiurge de l’univers, qui est, selon lui, celui qui a été, est, et sera (pouvant se traduire en hébreu par le nom de Jéhovah). C’est après avoir été accusé de corrompre la jeunesse, par les croyances en d'autres divinités qu’il diffusait de manière orale, qu’il fut condamné à mort. La chronique de son procès, rédigée par Platon, est toujours étudiée en faculté de droit.

Pour en revenir à mon pacte avec le divin, je ne laisserai jamais l’empreinte sur mon âme d’un meurtre commis sur le royaume terrestre.

\begin{center}
***
\end{center}

Aristote nous dit dans "Les Politiques" que la tyrannie peut exister dans n’importe quel type de régime. Il est probable que tout régime, bon ou mauvais, possède toujours une pointe de tyrannie, même si nous vivions dans le meilleur des mondes. Même si parfois, le mal s’élève et s’étend dans des proportions alarmantes au sein des communautés humaines, avec pour exemple marquant le régime nazi, ces phénomènes sont tout à fait cycliques, et le bien et l’équité finissent toujours par se restaurer par la force des choses ou par l’apparition d’un individu hors du commun, parfois élevé au rang de prophète ou de Messie.

La traduction du mot grec "apocalypse" est "révélation", et des déclins les plus funestes renaissent de nouvelles lois, de nouvelles mœurs, une nouvelle éthique. En clair, une nouvelle religion ou un enrichissement de l’ancienne religion plus adapté, où l’homme a su tirer les conséquences de ses erreurs passées.

Personne ne peut anticiper les catastrophes que l’humanité traversera, mais les périodes les plus meurtrières ont la particularité de cumuler quatre fléaux pouvant être assimilés aux cavaliers de l'apocalypse (Mort, Famine, Guerre et Conquête) soit:

\begin{itemize}
\item la guerre,
\item la famine,
\item les maladies,
\item les bêtes sauvages.
\end{itemize}

Quel(s) autre(s) fléau(x) pourrait(ent) être plus dévastateur(s) ?

Dans le cas de guerre, de maladies et de famine trop importantes, l’homme n’a plus les moyens logistiques pour dominer les animaux de la terre, et c’est à ce moment-là que les bêtes sauvages peuvent représenter un fléau important. Ces événements appartiennent à la nature du monde et poussent l’homme à l’avilissement, autant que l’avilissement des hommes alimente ces fléaux. Dans le cadre de l’eschatologie juive, chrétienne et musulmane, une apocalypse finale, sans précédent, où absolument tout est révélé à l’homme, porte avec elle un millénaire de paix où la science, la technique et le bien-être humain n'ont encore jamais été égalés.

Je crois possible cette apocalypse. Constatons-nous alors, une fois tout ce malheur terrestre passé, que les dix commandements sont totalement fondamentaux ? Trouverons-nous des commandements encore plus divins, plus adaptés, et encore plus porteurs de vitalité et de prestige civilisationnel ? Le christianisme sera-t-il triomphant ?

J’ai tout de même la nette impression que l’humanité ne trouvera pas mieux que les dix commandements de l’Ancien Testament, car des commandements meilleurs ne peuvent pas exister, à mon sens.

Que je voie ces fléaux arriver de mon vivant ou non, et quelle que soit la cause de la mort de mon corps, je souhaite avoir l’opportunité de laisser à mon âme une belle empreinte de mon passage terrestre. À l’instar des fléaux qui accentuent le péché et des péchés qui alimentent les fléaux, mon passage terrestre laissera une empreinte sur mon âme, tout comme mon âme laissera une empreinte sur le monde matériel (cf. chapitre 10). Mais une empreinte peu marquante, autant qu’une empreinte durable sur le matériel, finit toujours par s’éteindre de par la nature corruptible du monde matériel. La durabilité de l’empreinte que je laisserai au matériel m’importe peu. De ceux qui ont laissé une forte empreinte, nous ne retenons pas que les plus nobles d’entre nous, et l’histoire référence autant de monstres que de sauveurs.

Les Vikings avaient raison de croire que toute âme poursuivait ses combats sur le monde terrestre, y compris de manière post-mortem. Nous laissons inévitablement une empreinte spirituelle et matérielle sur le royaume terrestre, mais je crois que seule l’empreinte du matériel sur le spirituel est réellement éternelle.

\begin{center}
***
\end{center}

Mais ensuite, de quoi sera faite cette vie sans corps ni sensations corporelles ? Quelles sont les lois du monde immatériel ? Quelles seront alors les exigences de Dieu pour chacun de nous ? Les bénédictions ou les châtiments qui nous seront infligés seront-ils uniquement ceux induits par l’empreinte laissée par notre passage terrestre ? J’ai foi en une équité absolue des lois divines, et je suis donc convaincu que la rétribution des actes de chacun sera juste de manière tout aussi absolue.

N’ayant aucun point de vue matérialiste, l’univers est donc, pour ma part, composé de parties matérielles et immatérielles. L’homme, les bêtes, les végétaux et toute forme de vie sont composés de ces deux aspects (âme et corps). Par croyance, je pense que la partie immatérielle de la vie subsiste à la mort physique du fait de la nature incorruptible de l'immatériel.

Une vie après la mort ne serait alors pas destinée uniquement à l’homme, mais à toutes les espèces vivantes. Ce même principe d’empreinte des actions terrestres sur une âme immortelle s’applique alors à tout le vivant. Dieu doit cependant avoir des exigences diverses en fonction de ces espèces, puisque les comportements qu’elles doivent adopter pour assurer leur subsistance sont souvent totalement différents.

Il se pourrait que certaines actions menées sur terre soient abominables aux yeux de Dieu, au point qu’il décide de réduire à néant les âmes trop corrompues (c'est aussi une croyance chrétienne). Si ce Dieu est omnipotent, il en a la capacité. Mais le souhaite-t-il ? Le fera-t-il ? Nous laissera-t-il l’ultime choix de rester près de lui pour avoir la vie ou bien de s’en écarter totalement et d’infliger alors une mort définitive ?

Dans le catholicisme contemporain, il est considéré que l'enfer existe, mais qu'il n'y a pas nécessairement des âmes qui y sont damnées, puisque Dieu laisse toujours le choix, dont celui de revenir à lui. Il peut simplement être douloureux pour chacun de délaisser certains plaisirs terrestres dont nous serions trop tributaires afin d'être accueillis par Dieu à bras ouverts. Si le divin est bien édifié de la sorte, j'ai pour ambition d'y parvenir et de l'intégrer.