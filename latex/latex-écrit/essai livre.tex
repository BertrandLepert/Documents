\documentclass[20pt, twoside, openany]{extbook}
\usepackage[T1]{fontenc}
\usepackage[french]{babel}
\usepackage[utf8]{inputenc}


\usepackage{lmodern}
\usepackage{import}
\usepackage{titlesec}
\usepackage{titletoc}
\usepackage{graphicx}
\usepackage{hyperref}



\titlecontents{chapter}[0em]{\vspace{0.5ex}}{\footnotesize\contentslabel{1.4em}}{}{\hfill\footnotesize\contentspage}
\titlecontents{part}[0em]{\vspace{3ex}}{\bfseries\LARGE\contentslabel{1em}}{}{\hfill\small\bfseries\contentspage}


\usepackage[a4paper, total={13.8cm, 20cm}]{geometry}

\usepackage{fancyhdr}
\pagestyle{fancy}

\fancyhf{}
\fancyhead[LE,RO]{}
\fancyhead[RE]{\footnotesize{\leftmark}}
\fancyhead[LO]{\footnotesize{\leftmark}}
\fancyfoot[C]{\thepage}



\usepackage{pdfpages}
\usepackage{lineno}
%\usepackage{multicol}
%\usepackage{lastpage}



\renewcommand\headrulewidth{1pt}
\titleformat{\part}[display]{\bf\Large\centering}{  Livre \thepart :}{2pc}{}
\titleformat{\chapter}[display]{\fontsize{33pt}{33pt}\selectfont \bfseries}{Chapitre \thechapter}{2pc}{}


\makeatletter
\renewcommand{\maketitle}{%
    \vspace*{1cm}% ICI La taille que tu veux avant le titre
    \begin{center}%
    
    {\large
     %\lineskip .75em%
      \begin{tabular}[t]{c}%
       \small \@author
      \end{tabular}\par}%
      \vskip 5em%
    {\large \@date \par}%       % Set date in \large size.
 
    {{\LARGE \@title} \par}%
    \vskip 3em%
 	
 	\includegraphics[keepaspectratio=true, height = 3.3 cm]{arbre.jpg}
    
    \end{center}\par
    \vskip 3em%
}
\makeatother


\title{Essai sur l'existence et la nature du divin}
\author{Bertrand LEPERT}
\date{}




\begin{document}


\thispagestyle{empty}
\maketitle
\newpage
\thispagestyle{empty}


\vspace*{\stretch{1}}
\begin{center}
\begin{Large}
\textit{\`A Dieu, \\ 
à ma femme \\ 
\vspace{10pt}
et à ma famille proche.} 
\end{Large}
\end{center}
\vspace*{\stretch{1}}



\chapter*{Avant propos}
\thispagestyle{empty}

Ce livre a été écrit pour mettre en avant et partager certaines de mes croyances ontologiques et religieuses qui me sont particulièrement chères. Je suis conscient que certains passages peuvent sembler directifs, donc le lecteur est invité à ne pas adhérer aux propos qu'il juge trop insistants, incisifs ou même caduques.

Je me suis efforcé de rendre cet écrit le plus accessible possible, mais certains chapitres peuvent être difficiles à lire pour des lecteurs non initiés aux écrits philosophiques. Une compréhension incomplète de ces chapitres n’empêche pas la compréhension générale de l'ouvrage.

Cet essai est avant tout un partage, et son auteur reste très ouvert à la critique tant qu'elle reste constructive. Il est donc important que le lecteur garde un esprit critique à propos de cet écrit. Libre à lui d'adhérer ou non à l'ensemble ou à certaines parties des propos qui y figurent.

Je n'ai pas rédigé ce manuscrit dans le but de faire adhérer à ma pensée. Néanmoins, étant persuadé d'observer un déclin et une carence spirituelle dans nos sociétés, convaincu du danger que cela représente, et persuadé qu'aucune civilisation ne peut subsister sans religion, j’écris également dans l'espoir d'exercer une sorte de médecine que je juge nécessaire.

Le lecteur est donc chaleureusement invité à considérer à la fois la gravité du thème et la légèreté des différents livres et chapitres qui composent cet ouvrage.

\thispagestyle{empty}

\clearpage

\thispagestyle{empty}

\vspace*{\stretch{1}}
\begin{center}
\begin{quote}
\textbf{\textit{"La nature d'une civilisation, c'est ce qui s'agrège autour d'une religion."}}
\end{quote} 
\vfill\textbf{Paul Valéry.}
\end{center}
\vspace*{\stretch{1}}

\renewcommand{\contentsname}{Sommaire}
\tableofcontents



\part{Sur les principes de la création humaine et divine}

\chapter{Introduction : Des croyances religieuses et de ses dessins}

Certains savants définissent la religion par la croyance en une ou plusieurs divinités. Ainsi, les formes primitives de la religion, telles que l'animisme, ne sont pas considérées comme des religions. Pour parler de religion en des termes plus inclusifs, nous préférons la définir comme un ensemble de croyances considérées comme absolues, voire transcendantes. De cette manière, les religions primitives mais aussi l'athéisme occidental (possédant même cosmogonie, eschatologie et ontologie) peuvent être considérées comme des religions à l'instar des systèmes de croyances déistes. Il est à noter que tous ces systèmes de croyances sont vivement débattus et contestés, tout n'étant qu'une question de point de vue et de sensibilité. De plus, aucune religion encore pratiquée n'est figée, mais la plus sujette à des changements rapides reste l’athéisme occidental, car très appuyé sur des disciplines savantes modernes qui connaissent des évolutions rapides.

Ceci étant, tout individu, croyant ou non en l’existence du divin, possède donc un système de croyances. Ces croyances, exactes ou erronées, lui servent à rationaliser et orienter ses choix, à accomplir ses objectifs, et, au final, si ses actions sont adaptées, à atteindre les objets de ses désirs.

Ainsi, cet ensemble de croyances peut donc tout à fait être assimilé à une religion. La nature religieuse de l’être humain est donc indéniable et il lui est impossible d’y faire abstraction. En d’autres termes, chez l’homme, s'abstenir de toute religion lui est inaccessible.

Les individus se considérant comme non religieux sont donc dans la simple méconnaissance de leur caractère religieux, bien qu’ils n’ignorent en aucun cas posséder des croyances. Les croyances des plus hauts degrés (axiomatiques) servent à l'homme comme fondations lui permettant des raisonnements logiques et ainsi d’accéder à des croyances d’un ordre moins élevé. Ce sont les systèmes de croyances les plus exacts qui, employés à des actions concrètes, mènent aux desseins les plus adaptés. Mais si les croyances axiomatiques sont erronées, les raisonnements qui en découlent le sont aussi, et les desseins humains sont voués à l'échec.

Le succès des actions humaines, parfois grandioses, qu’elles soient individuelles ou collectives, reste, il me semble, le but premier de toute religion. Pour ma part, la faille de l'athéisme occidental réside dans le fait que ce système de croyances ne préconise aucune mœurs ni éthique autre que celles de son passé chrétien, sur lequel il s'est bâti et qui est de plus en plus oublié et attaqué.

\begin{center}
***
\end{center}

Si des croyances fausses sont enseignées délibérément à un ou plusieurs individus, c’est pour les contraindre à exécuter des actions que le menteur pense être la solution pour l'obtention de ses désirs. Le menteur peut également croire en des notions erronées qui lui ont été inculquées par un autre menteur ou un ignorant.

Manipuler un individu ou une population consiste donc à lui inculquer de fausses notions afin de lui faire exécuter les actions voulues par le manipulateur (si l’individu est convaincu qu’il y a un lion dehors, il ne sortira pas de sa maison). Bien sûr, plus le manipulateur est habile, plus le manipulé devient sa marionnette. Cela vaut autant pour un individu isolé que pour un phénomène de groupe plus ou moins conséquent.

Il y a manipulation lorsqu'un ou plusieurs individus inculquent à un ou plusieurs autres individus des notions soit habilement lacunaires (mensonge par omission), soit fausses afin de faire exécuter par le parti manipulé des actions qui arrangent le parti manipulateur.

Le double tranchant de la manipulation réside dans le fait qu'il reste possible que l'individu manipulé cherche à assimiler d'autres individus, que le manipulateur aurait voulu épargner.

Dans ce cas de figure, lorsqu'il s'agit de systèmes de croyances complets, il n'est plus question de religions, mais de sectes.


\begin{center}
\begin{quote}
\textit{"laissons au peuple le soin de croire que la science va réellement au fond choses."} \end{quote} Friedrich Nietzsche.
\end{center}

Ayant bénéficié d'un cursus universitaire scientifique, qui m'a conduit à travailler dans la physique et l'informatique, j'ai été surpris par le nombre d'ingénieurs et de chercheurs rencontrés qui vivent dans cette fausse croyance.

\chapter{Des principes véritables aux raisonnements qui découlent de ces lois axiomatiques}

La physique s’attache à déterminer des lois parfaites de phénomènes quantifiables et matériels qui s'observent dans la nature. Par conséquent, ces lois prennent obligatoirement racine dans des constats empiriques, car, en premier lieu, seuls nos sens et le jugement porté sur nos sensations nous permettent d’interpréter les phénomènes naturels. Sans sensations et sans capacité à les juger, nous n’aurions d’ailleurs aucune conscience de ces phénomènes.

Concernant les phénomènes microscopiques, des lois ont pu être déterminées, soit parce que ces phénomènes se manifestent à l’échelle macroscopique, soit grâce à la création d’instruments permettant d’observer puis d’interpréter ce qui se passe à cette échelle (comme, par exemple, avec l'accélérateur de particules).

Plus une loi découverte est fondamentale, plus elle est générale et englobe des lois caractérisant des phénomènes sous-jacents. Les lois les plus fondamentales ont donc une nature principielle, ce qui implique qu’elles sont impossibles à démontrer et sont donc utilisées comme axiomes. Les lois fondamentales sont donc toujours au plus proche des constats empiriques ou métaphysiques et servent de socle à des développements théoriques plus ciblés.

Par exemple, les trois lois de Newton, qui sont axiomatiques, sont utilisées comme principes fondamentaux lorsqu’il s'agit de phénomènes mécaniques et partent du constat que les forces exercées sur des objets matériels déterminent leurs mouvements. L’existence même du mouvement et des forces est constatée empiriquement, avec l'argument qu’aucun individu sain d’esprit ne conteste leur existence.

Le principe variationnel de moindre action, applicable à la mécanique et utilisé dans de nombreux domaines de la physique tels que l’optique, la thermique ou l’électromagnétisme, permet également de déterminer une trajectoire lorsque l’action employée à la produire est une donnée connue, et inversement. Ce principe est une définition contenant deux phrases littérales et est admis comme un phénomène naturel général. À partir de cette définition, il est alors possible de développer les calculs nécessaires à la quantification du phénomène.

Le principe de moindre action a été défini par Pierre Louis Moreau de Maupertuis au XVIIIe siècle:

\begin{center}
\begin{quote}
"L’Action est proportionnelle au produit de la masse par la vitesse et par l’espace. Maintenant, voici ce principe, si sage, si digne de l’Être suprême : lorsqu’il arrive quelque changement dans la Nature, la quantité d’Action employée pour ce changement est toujours la plus petite qu’il soit possible."
\end{quote}

\end{center}

Cette loi est donc une considération métaphysique, en premier lieu indépendante de tout problème physique.

C’est à partir de ces grandes lois axiomatiques que tout problème physique est considéré et résolu par une formulation mathématique. Il est arrivé dans l’histoire de la physique que certains principes découverts de manière empirique aient été justifiés par des principes plus généraux et se soient donc vu attribuer une démonstration théorique.

Je crois, comme Einstein le pensait, en une loi physique unique justifiant par le calcul théorique toutes les autres lois et les grands principes dont elles découlent. En revanche, je pense qu’il est impératif que cette loi soit un axiome indémontrable sur lequel s'appuient tous les phénomènes physiques.

Toutes les disciplines scientifiques fondamentales de la physique ou des mathématiques admettent des axiomes considérés comme indémontrables, émanant de considérations métaphysiques, et la découverte d’une loi unique pourrait les justifier par le calcul.

On voit donc que la découverte de cette loi unique reviendrait à comprendre parfaitement l’interaction divine, ou du moins le principe créateur et ordonnateur de l’univers, dans le cas où celui-ci n’a ni volonté ni intelligence. Pour ma part, je pense que toute création nécessite une volonté de créer et une intelligence organisatrice permettant de concrétiser cette action. Je trouve qu’il est suffisamment évident qu’il n’y a aucune raison pour que des choses se créent sans raison(s), et que si ces choses ont une raison d’être, c’est avant tout parce qu’elles ont été raisonnées.

Dans son ouvrage "De gravitatione", son auteur, Isaac Newton explique: 

\begin{center}
\begin{quote}
"On ne trouve guère d’autre cause à l’athéisme que cette notion de corps en tant que doté d’une réalité en soi, complète, absolue et indépendante."
\end{quote}
\end{center}


\chapter{De l’impossibilité de l’engendrement de l’univers par l’action seule du hasard}

Le hasard est un concept abstrait qui est employé pour déterminer le comportement d’un phénomène dont l’aspect déterministe reste incompris. Les lois de probabilité sont donc utilisées pour mieux comprendre les effets liés à une cause en partie ignorée. Un chiffre obtenu par un jet de dés, tout comme une combinaison gagnante au loto, obéissent en réalité aux lois physiques de la mécanique. Dans ces deux cas de figure, les forces exercées sur les dés comme sur les boules, et la géométrie de ces objets, si elles étaient parfaitement connues, rendraient tout à fait déterministes le résultat de ces tirages.

Connaissant tous les éléments du problème, les lois de Newton renverraient le résultat des tirages de manière systématique. Dans le cas où l’univers n’obéirait qu’à des lois physiques, aucune action ne renverrait un effet dû au hasard. Si l’on peut penser que tout est déterministe, il en résulte que le hasard n’est pas un phénomène ayant une existence concrète (au moins en ce qui concerne les phénomènes matériels).

Pour ceux qui croient, comme moi, au libre arbitre, cela implique que l’univers n’est pas totalement déterministe, puisque nous exécutons des choix impossibles à prévoir par des lois physiques qui ne sont pas non plus dus au hasard. De mon point de vue, l’univers n’est donc pas totalement déterministe, mais le hasard n’a cependant pas d’existence concrète. Si nos actions ne sont pas physiquement déterministes, cela est dû à notre capacité à effectuer des choix conséquents du fait que nous sommes des êtres conscients.

Pour les personnes croyant que le hasard possède une existence concrète, il reste tout de même impossible que la cause première de la création de l’univers provienne d’une probabilité mathématique. En première objection, tout concept (abstrait ou concret) nécessite d’être créé. Dans ce cas, l’univers ne peut être engendré par le hasard, car il y a alors nécessité d’un concept plus général devant engendrer lui-même l’existence du hasard. Il me semble qu’on ne peut pas raisonnablement penser que le hasard émane de lui-même ou a été engendré par lui-même.

En seconde objection, il est évident que le hasard, s’il existe, est un phénomène agissant sur des éléments préalablement présents et influençant leur comportement. Le hasard est donc tout à fait stérile s’il n’a aucun objet sur lequel exercer son action. Celui-ci ne peut qu’agir sur des objets déjà préexistants, ce qui implique qu’il ne peut pas créer ex nihilo ! Le hasard ne peut donc être un principe créateur.

Le hasard reste alors inapte à engendrer quoi que ce soit. Si celui-ci existait, il ne pourrait qu'avoir une influence sur des phénomènes impliquant des étants déjà existants. Pour un scientifique, les lois de la physique (même quantique) sont tout à fait déterministes, le hasard n'existe pas dans l'univers et n'est qu'un outil mathématique permettant d'estimer au mieux le résultat d'un problème lorsque certaines données sont inconnues.

Cependant, la physique ne s'attache qu'à traiter de la matière et de son mouvement. Cette discipline ne permet donc qu'une considération purement matérialiste de la nature et ne peut pas tout englober dans le cas où l'on croit que l'univers n'est pas uniquement composé de matière.

Si la physique vient infirmer l'existence du hasard en ce qui concerne la matière, il reste indéterminé que celui-ci ait ou non une influence sur des étants immatériels.

Nous pouvons alors tout de même citer une objection d'Albert Einstein sur la question du hasard dans la physique quantique:

\begin{quote}
"Dieu ne joue pas avec les dés."
\end{quote}

En d'autres termes, même si je crois en l'existence de phénomènes immatériels, je ne crois pas que le hasard puisse les influencer de quelque manière que ce soit. Donc, de mon point de vue, le hasard n'existe pas.

\chapter{Démonstration philosophique de l’existence du principe créateur et mise en exergue de son caractère intemporel}

Commençons par définir le néant comme l'absence de tout étant. Le néant étant de toute évidence une chose en soi, sa propre absence est incluse dans sa définition même. C’est la raison de son inexistence concrète. Le néant étant un concept, un concept n'est pas une absence de tout, c'est déjà quelque chose. Dans cet univers hypothétique, le néant en tant que concept, est-il alors omniprésent ou absolument inexistant ? Il semble bien que de son omniprésence suivrait son absence dès lors qu'il y aurait engendrement. Si l’on part de la genèse de l’univers avec un commencement où rien n’existe, ce serait le néant lui-même qui aurait pour charge de générer l’existence de tout étant. Ce néant serait-il alors anéanti pour autant ? Il est vraisemblablement impossible d’effectuer l’exercice dans le sens inverse, puisque la présence d’une ou plusieurs choses en soi a pour effet de rendre caduque l’hypothèse d’un néant omniprésent. Ce néant hypothétique devrait alors être à l’origine de tous les univers (matériels et immatériels), cohabiter avec toutes les choses puisqu'elles en seraient toutes issues. Mais au sens strict, l’absence de toute chose inclurait l’absence d’un néant en tant que concept. On voit là que l'inexistant est impossible en tous lieux et à tous moments, et que pour qu'il y ait de l'existence, il n’est plus possible de parler de néant originel, mais d’un principe créateur omniprésent spatialement et temporellement, ayant d'ailleurs créé ces deux concepts, et dont tout émerge.


\begin{center}
***
\end{center}

Mais alors, chaque étant est-il issu de ce principe un par un, de manière séparée, ou bien leur ensemble complet en est-il issu ? Il semble que les deux phénomènes cohabitent. En tout cas, un seul objet ou phénomène peut être issu de ce principe, tout comme un ensemble d’objets et/ou phénomènes de toutes natures. Si toute chose est générée ex nihilo en une seule fois, on peut alors considérer que l’univers est un et indivisible. À l’inverse, si les étants sont extraits un à un de ce principe, on peut envisager un univers fragmenté et inclure la possibilité d’isoler certains objets et phénomènes. Dans le premier cas, l’univers serait continu, tandis que dans l’autre, il serait discret, même s'il contenait une infinité d’éléments. La question (qui remonte à l’antiquité grecque) de savoir si l’univers est continu ou discret pourrait se résumer à cette approche ambivalente des objets et phénomènes issus du principe créateur, soit un à un, soit dans leur ensemble. Cela ne serait donc qu’une question de point de vue, dépendant de la considération de ses observateurs.

\begin{center}
***
\end{center}

Discontinuité dans la genèse d’un univers ? Absence de tout à l’origine des temps ($t = 0$), puis présence d'étants dès lors que le temps dépasse son origine d'une valeur infinitésimale ?

Que nenni, les objets et phénomènes sont issus du principe créateur en raison de l’impossibilité de l’existence du néant, telle qu’elle a été définie précédemment. On peut alors conclure qu'au minimum, le principe créateur a toujours été et est. Sera-t-il ? Est-il immortel, incorruptible ? Oui, puisqu’un retour à l’absence de toute chose impliquerait un retour au néant qui, comme nous l’avons vu, ne peut exister. Ce néant impossible est donc bien concret puisqu’il définit au minimum l’existence d’un concept. Il faut alors en conclure que le principe créateur à l’origine de l’univers a été, est, et sera. Soit traduit en hébreu, Jéhovah, un des noms employés dans les textes bibliques pour désigner Dieu.


\begin{center}
***
\end{center}

Il est admis que la genèse d’une représentation d’un concept de l'univers possède une fin ainsi qu’un début : l’idée. Deux cas de figure peuvent être considérés : l’idée est juste ou erronée.

Si l’idée est juste, sa forme est donc conforme à la nature de l’univers. Le concept a été, est et sera (il y a transcendance), tandis que sa découverte par l'homme prend un début et une fin. Le début est l’établissement de l’Idée, et sa fin, son aboutissement.

Si l’idée est fausse, celle-ci conserve un début et une fin. Le concept qui en découle, en revanche, possède une nature contraire à celle de l’univers. Celui-ci n’a jamais été, n’est pas et ne sera potentiellement jamais. Si l’univers évolue, il est pourtant probable que cette idée devienne juste. Dans ce cas, le concept existe sur une durée donnée. Peut-on dire que cette idée a été, est et sera ?

Dans un concept sur l'univers, du moins, il est possible d’y ajouter l’idée d’évolution. Ainsi, des Idées peuvent être ou avoir été justes, ou le devenir ultérieurement. Cependant, ce ne sont pas des lois applicables à la totalité de l’univers mais seulement à des instants donnés. Ces lois sont bornées dans le temps. Il est aussi possible de déterminer des lois bornées dans l’espace, et à la fois dans le temps et l’espace.

Or une loi fondamentale, transcendante, régissant l'univers, même si celui-ci est évolutif, a été, est et sera puisqu’elle devra prendre en compte cette idée même d’évolution et ainsi régir des lois sous-jacentes qui peuvent être bornées dans le temps et l’espace.

Ainsi, les fondements d’un univers ont été, sont et seront à tout instant. Ceci invalide la possibilité évolutive des lois fondamentales d'un univers, et donc celle du principe créateur.

\begin{center}
***
\end{center}

Les lois de l'univers (fondamentales ou non) dans leur globalité restent inconnues de l'homme en l'état. Ce sont des univers conceptuels qui représentent des fragments de l'univers global (matériel et immatériel). Pourtant, l'univers entier interagit avec l'homme et vice versa. On peut, dans un cas plus général, dire qu’un fragment d'univers, s'il n'est pas isolé, agit sur l'univers dans sa totalité, qui lui-même agit sur ce fragment. Si un phénomène est mis en place dans l'univers matériel, deux cas peuvent donc être considérés:

\begin{itemize}
\item Le fragment concerné où se situe l’origine du phénomène interagit sur l’univers qui lui même interagit sur le fragment et ainsi de suite... Le phénomène est ininterrompu (exemple du larsen ou encore de la bombe atomique).
\item Le fragment interagit sur l’univers qui, quant à lui n’exerce aucune rétroaction sur le fragment concerné. Le phénomène est directement interrompu en dehors du fragment concerné.
\end{itemize}

Ces deux cas sont bien sûr idéalisés puisque même une explosion atomique finit par s’atténuer. Dans l'univers matériel, nous n'avons donc que des cas intermédiaires plus ou moins importants de ces deux phénomènes possibles. La raison réside dans la transduction énergétique. L’énergie se conserve mais se dissipe sous d’autres formes autres que celle du phénomène initial. Cette énergie est réutilisée pour donner lieu à d’autres phénomènes extérieurs au fragment d'univers considéré. Il est donc tout à fait possible d’émettre l’hypothèse d’une source unitaire et primordiale d’énergie se diffusant pour donner lieu à des phénomènes multiples, réalimentant eux-mêmes d’autres phénomènes dans une boucle infinie.

\begin{center}
***
\end{center}

Y a-t-il alors possibilité d’une source ponctuelle d’énergie à l’origine des temps ?

Nous avons démontré plus haut que l’univers n’a pas d’origine temporelle en raison de l'existence du principe créateur en tout temps. Cependant, il n’est pas impossible que le temps, élément de cet univers et induit par son principe créateur, ait une origine. Lorsqu’il ne s’agit pas des fondements de l’univers, l’évolution est possible, et le temps peut aussi passer de l’inexistant à l’existant (peut-être même à plusieurs reprises).
Soit deux possibilités:

\begin{itemize}
\item Le temps est un fondement de l’univers et n’a pas d’origine ( il a été, est et sera).
\item Le temps n’est pas un fondement de l’univers et a une origine, n’a pas toujours existé (contrairement à son principe créateur). 
\end{itemize}

Or le temps est mesurable et évolue, et seul son concept en tant qu’Idée est inchangeant. En tant que phénomène, le temps est sous-jacent au(x) fondement(s) de l’univers, puisque tout ce qui est évolutif fait partie des concepts qui ne sont pas fondamentaux à la genèse d’un univers.

L’énergie elle-même évolue, notamment en fonction du temps, se dissipant et générant d’autres phénomènes. La matière évolue par le mouvement, et la masse varie en raison des mouvements mêmes de la matière. Pourtant, les Idées de ces concepts sont invariantes.

On sait que matière, énergie, et masse sont trois concepts physiques qui se conservent et ne font qu’évoluer sur le plan matériel. Nous avons démontré que l’existence d’un néant impliquerait au minimum la correspondance à un concept possédant une existence concrète, ce qui est un non-sens. Cela ne peut donc qu’impliquer l’existence d’un étant invariant selon tous paramètres, et à l’origine de toutes choses (lois, phénomènes, matière, énergie, temps, etc.). N’étant pas soumis au temps, cet étant est incorruptible.

En résumé, tout évolue sauf le principe créateur : Dieu et ses Idées. Le mouvement, le temps, et la matière (intrinsèquement liés et évolutifs), restent toujours des constats empiriques utilisés comme axiomes par les physiciens, et sont donc indéniablement des axiomes démontrables si l'on possédait la connaissance de cette loi unique régissant l'univers évoquée dans le chapitre 1.

\chapter{De l'existence d'un Dieu monothéiste par l'extrapolation du principe de moindre action}

Commençons par rappeler le principe de moindre action tel qu'il a été défini par Pierre Louis Moreau de Maupertuis au XVIIIe siècle:

\begin{center}
\begin{quote}

« L'Action est proportionnelle au produit de la masse par la vitesse et par l'espace. Maintenant, voici ce principe, si sage, si digne de l'Être suprême : lorsqu'il arrive quelque changement dans la Nature, la quantité d'Action employée pour ce changement est toujours la plus petite qu'il soit possible. »

\end{quote}
\end{center}

Ce principe reste fondamental dans la physique et, dans le cadre strict de cette discipline, permet une approche énergétique des lois du mouvement des corps. Par vulgarisation, on peut dire que lorsqu'un mouvement opère, celui-ci emploie le minimum d'énergie possible à sa réalisation.

Puisqu'une action créatrice devrait nécessiter de l'énergie, sans doute elle-même créée en amont, il n'est pas déraisonnable d'émettre l'hypothèse de l'extrapolation de ce principe pour expliquer l'action créatrice de l'être suprême.

Créer un univers, même originel, puisque le hasard n'a pas d'existence concrète, n'est en rien dû à un "coup de chance" (cf. chapitre 3). Soit l'univers émane de lui-même, soit il a été créé, a évolué, et est, dans ce cas, soumis au changement. Or, s'il n'y a aucune raison à ce changement, c'est bien que l'univers change sans raison. S'il change sans raison, c'est qu'il n'a aucune raison de changer.

Pourtant, nous avons vu par le principe de moindre action que l'univers, lorsqu'une action est effectuée, tend à minimiser la dépense énergétique nécessaire. Aucune action superflue n'est alors effectuée.

Par conséquent, si l'univers n'avait aucune raison d'être en changement, ce changement serait superflu et n'aurait donc pas lieu. Cela montre que l'univers possède donc une raison à son changement. Il semblerait qu'une raison supérieure, exclue de l'univers, autorise ce changement ou mouvement, lui donne une raison, et, en ce qui concerne la matière, interdit toute autre action inutile ou superflue (pas de mouvement s'il n'y a pas de raison nécessaire).

Si le changement a donc une raison d'être, c'est bien qu'il a été raisonné par un Dieu, et l'hypothèse de plusieurs divinités serait d'ailleurs également superflue et donc invalide.

\chapter{De l’existence d’un repère spatial absolu de l’univers et des mouvements véritables}

Le mouvement des corps s’observe de façon relative par rapport à un lieu soit réellement fixe, soit considéré comme fixe. Cette notion est connue depuis l’antiquité, car Aristote, en traitant de la notion de lieu et de mouvement dans son ouvrage "Physique", démontre philosophiquement cette notion pouvant être considérée comme pré-relativiste.

Soit un train A à l’arrêt par rapport au repère terrestre et un train B suivant une trajectoire quelconque par rapport à ce même repère (ici considéré comme fixe). Dans ce cas, c’est le train B qui paraît en mouvement relativement au train A. De la même manière, si un observateur au repos dans le train B observe le train A et le décor, celui-ci verra ces éléments en mouvement, et s’il se considère comme fixe dans l’espace, il admettra que le train A et son décor sont animés d’une vitesse.

Dans l’histoire, les philosophes ont longtemps débattu de l’existence d’un repère spatial absolu. Si ce repère existe, cela implique qu’au moins un point de l’espace est réellement fixe de manière absolue et permettrait donc d’observer les mouvements tels qu’ils sont en réalité. Pour Aristote, un repère fixe existe bel et bien et est dû à l'implication d'un dieu qui crée le mouvement originel, moteur de tous les autres.

Descartes, cependant, ne concevait que des mouvements relatifs par rapport à des lieux et ne définissait d’ailleurs l’espace que par la présence d’objets plus ou moins éloignés les uns des autres. Newton défait par la suite cet élément de la doctrine cartésienne en apportant l’argument que faire tourner l’étendue du ciel étoilé autour de la Terre n’est pas le même phénomène que faire tourner la Terre sur elle-même et observer depuis celle-ci que le ciel est en rotation, car cela n’implique ni les mêmes grandeurs en termes de forces, ni les mêmes points d’application de ces forces.

Pour Newton, ce repère absolu existe donc bel et bien et peut être déterminé par la connaissance de toutes les forces appliquées à un objet de l’espace. Cette notion à priori juste renoue donc avec le point de vue aristotélicien d’un repère spatial absolu de l’univers, et nous pouvons alors observer que la Terre tourne réellement sur elle-même par rapport à ce repère absolu en mesurant la force de Coriolis.

Pour Newton, moi-même, et tout autre individu déiste et monothéiste, si une force semble provenir sans autre conséquence qu’elle n’existe que par elle-même, c’est que Dieu l'applique directement de par sa volonté, ou bien qu'une mise en mouvement primordiale est responsable de ces forces engendrant tous les autres. Bien qu'Einstein ait ajouté une conséquence nouvelle aux forces astronomiques par la déformation de l'espace-temps due à la masse des objets célestes, cela n’exclut pas une nature divine de ces phénomènes.

Mais les forces peuvent être la conséquence de phénomènes plus élevés se situant entre Dieu et elles-mêmes. Dans ce cas, l’action de Dieu la plus directe ne serait pas dans la création, le maintien, et le transfert des forces, mais dans quelque chose qui aurait notamment pour conséquence de créer des forces et du mouvement. On sait d’ailleurs que la présence de matière implique une attraction gravitationnelle, et, depuis Einstein, par la déformation du continuum espace-temps induite par sa masse. Donc, une mise en mouvement primordiale à l'origine des temps pourrait être due à la création originelle de matière.

Par analogie avec les êtres vivants, l’homme et l’animal peuvent mettre des objets en mouvement par la volonté. Mais cette volonté a d’abord pour effet d’actionner une mécanique corporelle analogue à celle d’un automate pour transmettre ses forces aux objets environnants. Une différence entre l’homme et les espèces animales est que l’homme possède une maîtrise supérieure de ce à quoi ces mouvements peuvent lui servir et détient donc une raison plus divine et raisonnable lui permettant d’exécuter ces actions.

Je me demande donc si Dieu maintient l'existence de forces et de mouvements par volonté directe ou si ses lois ont été établies préalablement et qu'aucune action de sa part n'est exécutée par la suite.

La mécanique classique, dans le cadre de cette discipline, a renvoyé Dieu à l'idée d'un horloger qui a établi ce système sans exercer d'interaction autre que sa création. Mais est-ce aussi la conséquence d'une volonté raisonnable à la présence de ces mouvements qui nous dépasse? Quel est le dessein de toute cette mise en mouvement?

Les êtres pensants, autres que Dieu, sont aussi les seuls à pouvoir influencer par la volonté, puis par l'action, ce mécanisme gigantesque qui, autrement, ne serait soumis qu'à la seule volonté divine. Dieu, ou le principe créateur, permet donc aux êtres vivants, et eux seuls, de mettre en mouvement ou de faire dévier, par l'exercice de leur libre arbitre, de manière consciente, des objets matériels dont les trajectoires seraient sinon uniquement dépendantes de(s) force(s) céleste(s) originelle(s).

\chapter{La trinité, l'un et le multiple}

En théorie des nombres, il faut d'abord considérer deux nombres si l'on veut engendrer tous les entiers par le calcul. Soit, par exemple, l'ensemble ${2,3}$ où le chiffre trois ne peut en aucun cas être engendré. Platon appelait ce chiffre le "chiffre sans père" car impossible à recréer sans le considérer directement. En théorie des nombres, pas de passage de l'un au multiple sans préalablement considérer le chiffre trois.

Que les nombres aient ou non une existence concrète reste, à mon sens, une grande question. Sont-ils une simple abstraction humaine ou existent-ils concrètement par acte de création extérieure en tant qu'immatériels?

Sans l’existence d'un démiurge, sans architecte de l'univers, les nombres ne peuvent exister qu'en tant qu'abstraction humaine dans le but de dénombrer et de donner du sens à des étants qui en seraient dépourvus. En revanche, si Dieu existe, et bien qu'il reste possible que celui-ci ait établi sa création autrement que par l'utilisation de nombres, en utilisant d'autres règles, il reste probable que les nombres aient une existence concrète en tant qu'immatériels et que, par exemple, l'espace tridimensionnel dans lequel nous sommes projetés possède son unité de mesure divine. Peu probable dans ce cas que le mètre ou le pied aient été employés.

Dans ce cas précis, pour des raisons de justesse conceptuelle, il serait commode de déterminer ces unités et règles de mesure et de les employer dans nos calculs scientifiques.

\begin{center}
\begin{quote}
\textit{"Dieu c'est la juste mesure des choses."}
\end{quote} Platon.
\end{center}

Autant donc s'employer à mesurer avec justesse si l'on veut mieux comprendre les phénomènes naturels. Pour ma part, je crois que l'architecte a créé et utilisé les nombres dans sa création. À ce jour, l'homme n'a pas trouvé de système plus approprié que le système métrique pour les calculs physiques; produit de la Révolution française, les savants qui se sont attelés à son élaboration considéraient qu'il serait employé par les scientifiques du monde entier en raison de sa supériorité. De nos jours, ce système reste indétrônable. Les sept unités physiques primordiales du système métrique, permettant le calcul de toutes les autres, sont : le mètre, le kilogramme, la seconde, l'ampère, le kelvin, la mole et la candela. J'ai personnellement l’impression qu'il devrait être possible de quantifier tous les phénomènes : courants électriques, température, quantité de matière et lumière, uniquement avec des unités de distance, de poids et de temps. Soit trois unités au total.


\begin{center}
***
\end{center}

Lorsqu'il s'agit de concepts matériels, observables dans la nature, tels qu'une couleur unique, celle-ci peut toujours se décomposer en trois couleurs primaires, et nous avons alors de nouveau un exemple du passage de l'un au multiple par la présence du chiffre trois. L'ensemble infini des couleurs peut être recréé à partir de trois autres couleurs primitives.

Ce concept métaphysique se retrouve aussi dans le principe trinitaire de la religion chrétienne et vient justifier le passage de l'un au multiple dans la création divine unique.

Au moins une autre mythologie intégrant ce savoir est la religion scandinave où le symbole du Valknut, représentant trois triangles entrelacés, et donc trois figures à trois côtés, donne une indication sur un degré estimable en termes de connaissance métaphysique qu'a pu établir ce peuple, puisque apparemment, un principe créateur basé sur des multiples de trois était considéré.


\begin{center}
***
\end{center}

Le concept de dyade inventé par les philosophes grecs met en avant le fait que certains concepts subjectifs existent de manière intrinsèquement liée à leurs contraires. L'un ne peut exister sans l'autre, comme le lourd et le léger, le dur et le mou, ou encore le clair et le sombre. Mais cela ne s'applique qu'à des considérations purement subjectives, puisque le lourd et le léger n'existent que relativement à la valeur de la masse de l'objet, le dur et le mou par la raideur, et le clair et le sombre par la quantité de photons réfléchie par la surface observée.

Enfin, la monade, conceptualisée par Pythagore, renvoie à l'unicité cohérente de l'univers incluant toute chose. Cette idée peut aussi bien renvoyer à Dieu qu'à un univers englobant tout et émanant de lui-même, mais aussi à la notion de panthéisme, où tout, y compris la matière, fait partie de Dieu.

Alors quoi ? Si, à ce moment-là, tout est interprétable comme un langage, voire simplement une harmonie numérique, l'univers aurait-il été composé sur une rythmique ternaire ?

Il semblerait que chaque phénomène naturel possède cet aspect ternaire ou trinitaire. En mécanique, force, masse et mouvement sont liés. En électromagnétisme, il en est de même pour le courant, le champ magnétique, et la force induite. On peut encore citer la célèbre formule fondamentale d’Einstein $E = mc^2$. Toute loi fondamentale de la physique ne fait intervenir que trois termes.

Après tout, la Bible enseigne que Dieu est une parole et que toute parole possède sa métrique.

\part{De la nature de l'âme, de la présence divine et de ses lois}

\chapter{Sur l’absence de silence de Dieu}

Là où un panthéiste pourrait considérer observer Dieu en toute chose, d'autres, comme moi, considèrent que le matériel et l'immatériel sont distincts de Dieu et résultent de sa création.

Dans ces deux cas, il est toujours possible d'acquérir des connaissances sur le divin (ou, dans le cas d'un athée, sur un univers émanant de lui-même, dépossédé de sens).

Si Dieu était totalement silencieux, nous n'observerions ni matière, ni n'entendrions de sons, et ainsi de suite. Nous serions totalement coupés de son univers. Un silence total de Dieu impliquerait que nous soyons également coupés de nous-mêmes, et il nous faudrait alors être dépourvus de conscience de nous-mêmes et de notre environnement. La simple observation et les interprétations de l'univers renseignent alors sur les mystères du divin.

Pour autant, religions et sciences, sur des plages temporelles millénaires, ont toujours évolué vers plus de justesse puisque l'homme, au fil des âges, a toujours pu acquérir une compréhension plus profonde de son environnement et des lois qui le régissent.

Au moins un indicateur incontestable est l'évolution de ces sciences et l'élaboration d'outils et de machines de plus en plus sophistiquées, permettant une meilleure maîtrise de notre environnement. Ces outils, qui à leur tour (par exemple la lentille convergente), ont permis, par leur emploi, d'augmenter notre savoir, d'enrichir encore notre compréhension du monde, d'élaborer de nouveaux outils, et ainsi de suite.

La création de ces outils, résultant en premier lieu d'une observation des lois de l'univers, a été permise par des évolutions et des bouleversements paradigmatiques, métaphysiques et des recherches fondamentales. La construction de machines n'étant qu'une discipline de l'ingénierie s'appuyant sur des recherches savantes abstraites (et, en premier degré, la compréhension de la création divine) donne l'impression d'être tout à fait inutile lorsqu'il s'agit de sciences si elles ne permettent pas la construction d'objets utiles.

Nous pouvons émettre le même raisonnement concernant les découvertes en termes d'éthique et de mœurs toujours plus adaptées et améliorant le bien-être humain. Cependant, pour un athée, il me semble que ces choses sont de pure abstraction qui n'ont pas de sens concret.

En résumé, Dieu ou l'univers nous communique ses savoirs en nous permettant l'observation, l'interprétation, la réflexion, et la construction matérielle et spirituelle. Dieu est avant tout une recherche humaine qui a certainement débuté dès son apparition. Cette recherche est donc millénaire et ses avancées sont très lentes. Nous n'apprendrons sans doute pas grand-chose de plus de notre vivant, mais ses mystères se dévoilent siècle après siècle. Dieu n'est donc pas silencieux car il nous autorise à minima à observer sa création et à la comprendre.



\chapter{Perception religieuse de la nature de l'univers}

Pour un individu déiste, tout comme pour un individu athée, le constat le plus élémentaire que l’on puisse faire vis-à-vis de notre rapport à l’univers est que nous sommes tous soumis, collectivement comme individuellement, à un ensemble de lois et de contraintes dues à notre environnement.

Les lois fondamentales de la physique ne peuvent être violées et, à titre d’exemple, on ne peut pas décider d’atterrir sur la Lune par un simple saut en hauteur. Il faut s’équiper du matériel adéquat.

D’autres lois, pourtant plus permissives, peuvent être contrées par l’exercice de notre libre arbitre. Ces lois ont, il me semble, une existence bien concrète, puisque autoriser le meurtre et le vol, par exemple, serait une catastrophe pour notre vie collective. Si ces crimes se généralisaient pour ne plus être réduits à des épiphénomènes, alors notre environnement ne serait plus propice à notre simple survie.

On peut dans ce cas, faire le constat que certaines actions humaines sont plus adaptées que d’autres pour la vitalité d’une civilisation et des individus qui la composent. Le tout reste alors de comprendre quelles sont les meilleures règles à adopter, d’abord pour notre survie, puis pour notre bien-être.

Pour certains, il s’agit de lois naturelles, et on peut souvent entendre que ces choses proviennent de "la nature". Pour ma part, il s’agit là d’une vision réductrice de cet ensemble de phénomènes, car je ne peux pas m’empêcher de m’interroger sur la nature de cette "nature". D’où et de quoi proviennent ces lois ? Qu’est-ce qui les maintient ? Comment ont-elles été créées et pour quelle(s) raison(s) ? Comme pour la physique, ne s'agirait-il pas de lois divines ?

Il est toujours possible de songer que le monde, l’univers dans lequel nous baignons, n’est qu’une illusion, est irréel, et n’a aucune existence concrète. Certains contemporains le pensent et écrivent sur ce sujet, mais cette interprétation figure aussi dans des croyances beaucoup plus anciennes, comme le maya des hindouistes et des bouddhistes. Cependant, une illusion doit être causée par au moins un élément ayant une existence concrète. Si cet élément disparaît, il en va de même pour l’illusion créée.

Il s’agit d’un raisonnement métaphysique simple qui dit que si absolument tout est irréel, alors tout est réellement irréel, et nous nous heurtons alors à un paradoxe. Cette ontologie religieuse est une des différences majeures avec celles apportées par les religions helléniques et judéo-chrétiennes qui, quant à elles, admettent une création et une existence concrète de l’univers, tandis que les religions issues du brahmanisme admettent un univers irréel produit par le rêve de Dieu.

Il n’y a alors que trois solutions :

\begin{itemize}
\item du point de vue de l'athéisme occidental, l’univers a un jour émané du néant et évolué pour prendre la forme dans laquelle nous le trouvons aujourd’hui.
\item l’univers peut aussi être potentiellement imaginé ou rêvé par quelque chose qui ne peut alors qu’être une entité pensante. Soit l’univers n’existe pas concrètement.
\item Il est enfin possible que l’univers ait été créé ou engendré par une entité et nous vivons dans un environnement matériel réel qui a nécessairement été pensé et créé.
\end{itemize}

Étant convaincu que nous possédons un libre arbitre, j’adhère personnellement à la troisième hypothèse puisque si l’univers est imaginé, les actions humaines le sont aussi, et nous vivons alors dans l’illusion de posséder un libre arbitre. De plus, je considère comme tout à fait juste le célèbre principe de René Descartes : \begin{quote}\textit{"Je pense, donc je suis"}\end{quote}, qui implique alors que tout être pensant existe bel et bien en tant qu’étant. Si les êtres vivants étaient eux aussi imaginés, ceux-ci ne posséderaient donc pas la pensée. Enfin, nous avons vu au chapitre 4 que l'univers ne peut émerger du néant, et un principe créateur non engendré et donc non préalablement raisonné me semble aberrant.

Pourtant, nous n’observons et ne pensons pas l’univers tel qu’il est réellement, sinon nous en aurions tous une connaissance parfaite et toutes les sciences deviendraient inutiles. L’association de nos observations sensorielles et de notre capacité d’abstraction serait suffisante pour comprendre absolument tout.

Il faut alors s’appuyer sur le mythe de la caverne de Platon, très connu des académiciens. Voici donc un extrait du septième livre de "La République" de Platon:

\begin{quote}
\textit{"Eh bien, après cela, dis-je, compare notre nature, considérée sous l’angle de l’éducation et de l’absence d’éducation, à la situation suivante ; Représente-toi des hommes dans une sorte d’habitation souterraine en forme de caverne. Cette habitation possède une entrée disposée en longueur, remontant de bas en haut tout le long de la caverne vers la lumière. Les hommes sont dans cette grotte depuis l’enfance, les jambes et le cou ligotés de telle sorte qu’ils restent sur place et ne peuvent regarder que ce qui se trouve devant eux, incapables de tourner la tête à cause de leurs liens.
Représente-toi la lumière d’un feu qui brûle sur une hauteur loin derrière eux et, entre le feu et les hommes enchaînés, un chemin sur la hauteur, le long duquel tu peux voir l’élévation d’un petit mur, du genre de ces cloisons qu’on trouve chez les monteurs de marionnettes et qu’ils érigent pour les séparer des gens. par-dessus
ces cloisons, ils montrent leurs merveilles.}
\begin{itemize}
\item \textit{Je vois, dit-il.}
\item \textit{Imagine aussi, le long de ce muret, des hommes qui portent toutes sortes d’objets fabriqués qui dépassent le muret, des statues d’hommes et d’autres animaux, façonnés en pierre, en bois et en toute espèce de matériau. Parmi ces porteurs, c’est bien normal, certains parlent, d’autres se taisent.}
\item \textit{Tu décris là, dit-il, une image étrange et de bien étranges prisonniers.}
\item \textit{Ils sont semblables à nous, dis-je. Pour commencer, crois-tu en effet que de tels hommes auraient pu voir quoi que ce soit d’autre, d’eux-même et les uns des autres, si ce n’est les ombres qui se projettent, sous l’effet du feu, sur la paroi de la grotte en face d’eux?}
\item \textit{Comment auraient-ils pu, dit-il, puisqu’ils ont été forcés leur vie durant de garder la tête immobile?}
\item \textit{qu’en est-il des objets transportés? N’est-ce pas la même chose?}
\item \textit{Bien sûr que si.}
\item \textit{Alors, s’ils avaient la possibilité de discuter les uns avec les autres, n’es-tu pas d’avis qu’ils considéreraient comme des êtres réels les choses qu’ils voient?}
\item \textit{Si, nécessairement."}
\end{itemize}
\end{quote}

Cette allégorie vient nous expliquer que notre condition humaine implique que nous observons et interprétons l’univers sous une forme déformée de la réalité, partiellement illusoire. Notre prison et nos chaînes qui nous empêchent d’observer et de comprendre parfaitement la réalité sont les limites de nos sens, de notre capacité d’interprétation, et de la qualité de nos raisonnements.

Il reste pourtant possible de s’émanciper partiellement et peut-être (qui sait ?) un jour totalement de cette infirmité. Nous pouvons alors citer Joseph Fourier:

\begin{quote}
\begin{center}
\textit{"Les Mathématiques sont une faculté de la raison humaine, destinée à suppléer à la brièveté de la vie et à l’imperfection des sens."}
\end{center}
\end{quote}

J'ajouterai qu'on pourrait en dire autant de toutes les sciences, théologie et philosophie comprises. Si les sciences permettaient un jour d’atteindre la compréhension et la maîtrise totales des phénomènes naturels, serait-ce réellement sage ? Je ne crois pas que l’homme soit assez mature pour tout contrôler avec une discipline infaillible. Une connaissance totale de l’univers implique certes une connaissance parfaite de l’éthique transcendante qui y serait à priori incluse, mais encore faut-il être en mesure de se l’imposer à soi-même de manière tout aussi parfaite. Ce serait une tâche herculéenne, même pour le plus vertueux des hommes, de par notre nature pécheresse.

Un chrétien convaincu vit aussi dans la conviction qu’à la fin des temps, tout sera révélé. La connaissance et la maîtrise de toute chose feraient pourtant de nous des individus omnipotents, et pour un chrétien, il ne peut y avoir plusieurs dieux. Une révélation totale implique alors que l’individu, ne pouvant devenir lui-même un dieu, fait alors partie intégrante du Dieu avec lequel il ne fait qu’un.

Bien que les théologiens du christianisme ne le pensent pas tous de la sorte, c'est le cas dans le catholicisme contemporain, et il est aussi possible d'y voir une analogie avec la pensée hindoue où, passé un certain nombre de réincarnations successives, les âmes retournent à l’âme primordiale de laquelle elles sont issues. Parfois, les religions s’influencent mutuellement. Il est toujours possible qu’elles soient toutes issues et dérivées d’une première et unique religion primitive. Après tout, au XIXe siècle, les historiens considéraient toujours un monothéisme originel qui s'est dénaturé après la chute de Babel et l'apparition du zoroastrisme, la religion la plus ancienne encore pratiquée aujourd'hui. Ce n'est plus une pensée dominante à notre époque, mais elle compte toujours des adhérents.

Peut-être que l’hindouisme n’a nullement influencé le christianisme, le débat est toujours ouvert. Dans tous les cas, les solutions plausibles à ces questions métaphysiques sont en nombre assez limité.

\chapter{L'âme et la matière}

Définissons la conscience d'un étant par le fait qu'il possède la connaissance de sa propre existence et, dans la plupart des cas, de ce qu'il perçoit comme étant son environnement de façon dissociée.

Une question importante, toujours en vigueur à notre époque, est celle de l'âme ou de la conscience engendrée ou non par la matière. En d'autres termes, une machine peut-elle être rendue consciente ? Sommes-nous nous-mêmes semblables à des machines conscientes grâce à une activité cérébrale qui n'est, en somme, que des électrons en mouvement ?

Pour le philosophe René Descartes, de par sa notion de dualisme, le corps de l'homme est semblable en tous points à une machine physique, mais sa pensée, sa capacité d'introspection sur ses raisonnements l'en différencient. Son âme est métaphysique.

Certains penseurs comme Karl Marx ou Thomas Hobbes ont émis l'assertion que la conscience émane directement de la matière en mouvement. Dans ce cas, il serait un jour possible d’insuffler une âme à une machine.

Cette question revient actuellement sur le devant de la scène, jusqu'à atteindre le grand public, avec les récentes avancées en matière d'intelligence artificielle. Pour autant, ces algorithmes simulant une forme d’intelligence ne sont en rien conscients. D'ailleurs, un ordinateur contrôlé par n'importe quel programme est aussi, en premier lieu, une mise en mouvement d'électrons dans des circuits.

Je réfute personnellement l'idée que la conscience puisse émaner de matière en mouvement, même complexe (si tant est que la nature comprenne la notion de complexité). Cela signifierait que le mouvement de matière, purement matériel par tautologie, peut engendrer des étants immatériels, puisque la conscience existe en tant qu'immatériel. Il manquerait donc des connaissances fondamentales sur la nature de la matière, car au vu des connaissances scientifiques contemporaines où toutes les lois physiques régissant le mouvement de matière sont considérées comme parfaitement connues, ce phénomène est impossible.

De plus, la physique nous enseigne justement que rien ne se crée, et selon la maxime célèbre d'Antoine de Lavoisier : \begin{quote}\textit{"Rien ne se perd, rien ne se crée, tout se transforme."}\end{quote}. Un mouvement de matière, engendrant une conscience, le tout sans transfert et donc perte de matière pour alimenter cette conscience, viole la physique la plus élémentaire.

Donc, de mon point de vue, la conscience est un immatériel non engendré par la matière et son mouvement. Nous possédons par conséquent une âme. Puisqu'il ne s'agit pas de matière et qu'un lieu est défini par de la matière dans un espace, la conscience, ou l'âme, contrairement à notre intuition première qui nous pousse à la situer dans notre tête, ne posséderait à priori pas de lieu.

Le propre des étants immatériels reste, semble-t-il, leur incorruptibilité, c'est pourquoi je crois fermement que toute conscience humaine ou animale survit à la mort de son enveloppe charnelle. Donc, pour ma part, l'âme doit être immortelle.

Cependant, je pense que ces mouvements d'électrons caractérisant l'activité cérébrale ont une utilité, car, d'une part, nous avons vu dans le chapitre 5 que rien n'est créé sans raison. Je subodore qu'à l'instar d'un supercalculateur dont la technologie nous dépasse encore, cette activité cérébrale est interprétée par notre conscience en ce qui concerne nos sens mais aussi nos facultés cognitives.

Puisque nous percevons ces éléments, il semblerait que, à minima, le long de notre vie terrestre, ce que nous sommes sur le plan matériel communique et interagit avec notre étant immatériel pour acquérir expérience et connaissance. L'inverse est aussi valable, puisque notre conscience nous pousse à exécuter des actions et à utiliser nos facultés cognitives. C'est de cette manière que le monde matériel laisse une empreinte sur notre âme, et que notre âme, lors de notre passage sur terre, laisse aussi son empreinte au monde matériel. Ceci est mon hypothèse.

\chapter{De la notion d'éthique dans la religion}

L'aspect éthique d'un individu découle d'ensembles de notions vues comme absolues. En effet, l’éthique ne peut pas être vue comme une notion absolue en elle-même, car elle nécessite de faire intervenir d’autres vérités plus élevées et est également confrontée aux perceptions émotives des individus. Le questionnement éthique et son établissement permettent d’obtenir une certaine conviction de discernement entre le bien et le mal.

C’est donc que les croyances de l’individu influent grandement sur son sens éthique. On peut déjà le constater dans les différentes mœurs qu'ont connues les peuples à travers l’histoire. Ces mœurs ont été forcément vues comme éthiques par ceux qui les pratiquaient et parfois comme abjectes par les peuples voisins. Pour un peuple occidental moderne, les guerres saintes (non exclusivement celles menées par l’Islam) sont vues comme des actions allant à l’encontre de leurs valeurs. Il en est bien sûr de même pour des notions encore plus barbares telles que les sacrifices humains rituels ou le cannibalisme. Dans le cadre de l'athéisme occidental, ce sont les valeurs humanistes qui sont considérées comme fondamentales et sur lesquelles l'éthique de ces sociétés est fondée.

\begin{center}
***
\end{center}

L’éthique d’un individu dépendra donc de son éducation religieuse (puisque nous considérons ici l'athéisme comme une religion) et de ses émotions et expériences propres. Son éducation a pour effet de modeler en partie ses émotions, tandis que ses émotions modèlent la perception qu’il a de son éducation et/ou de son attrait au divin.

On voit aussi que, dans le cas de l’éthique, on peut effectuer une dichotomie entre une éthique provenant d’une considération du divin, donc au-dessus des hommes (considérée comme transcendante pour le croyant), et une éthique théorisée par des penseurs humains sans lien avec le religieux. Dans le premier cas, l’individu n’ira (sauf en cas de reconnaissance d’un prophète) ni réinventer ni contester l’éthique qui lui a été enseignée. Dans le second cas, cette éthique peut être contestée, réinventée, remodelée, et cela, même à grande vitesse, lorsque la personne qui en a été instruite en éprouve le besoin ou la volonté. L’individu pourra aussi s’autoriser à réviser son éthique lorsqu’il estimera avoir dépassé son pédagogue.

Donc, dans le cas de religions basées sur des écrits millénaires, et bien que les interprétations des textes puissent être révisées, les évolutions en termes d'éthique sont beaucoup plus lentes que pour l'athéisme occidental.

Pour cette raison, dans le cas d’un individu ou d'un groupe athée, l’éthique peut subir des changements beaucoup plus brusques et rapides, car les notions les plus générales de leur éthique, et dont les autres découlent, sont tout à fait disposées à être renversées. De plus, autant un groupe d’individus de la même religion partagent au minimum les valeurs les plus hautes d’une éthique commune, autant des individus athées ayant fait varier trop rapidement et profondément leur éthique au gré de leurs expériences de vie seront aussi plus fragilisés dans leurs liens et actions communautaires, lorsque chacun d’entre eux adopte une éthique tout à fait personnelle et parfois non conforme à celle de ses congénères ou à l'éthique humaniste. Il en est de même lorsque plusieurs religions cohabitent au sein d'une même nation et que des tensions religieuses finissent par émerger en partie à cause de divergences éthiques. D'ailleurs, l'humanisme est inclusif avec toutes les religions, cherche à les assimiler et les faire cohabiter, ce qui peut constituer un potentiel danger.

\chapter{Des actions individuelles aux actions collectives}

Tandis que les religions primitives ne traitaient que de phénomènes collectifs sans s'attacher aux actions individuelles, depuis le judaïsme (et sans doute dans des religions plus anciennes), un Dieu caractérisant à la fois le personnel et le collectif est considéré. C'est à ce moment non daté de l'histoire qu'a émergé l'idée du caractère individuel et collectif des hommes.

Dans le chapitre précédent, nous avons traité l'éthique principalement sous l'angle des actions individuelles. Il est pourtant suffisamment simple de constater qu’une action individuelle se répercute sur le plan collectif dès qu’elle est observée. Le meurtre d’un individu atteint aussi la personne tuée et, par extension, ses proches ou une plus large partie de la population si l’acte est plus massivement diffusé. Il en est de même pour une action neutre ou bénéfique pour l’individu qui la reçoit. Les moindres échanges commerciaux peuvent même influencer l’économie à plus ou moins grande échelle.

Chaque action d’un individu unique sera donc perçue comme positive, neutre ou négative, éthique ou contraire à l'éthique, par chaque personne affectée. Cette action isolée influe donc à plus ou moins grande échelle dans la sphère collective.

Ces individus marqués par cette action peuvent choisir d’agir de manière passive, c’est-à-dire a minima, analyser avec plus ou moins d’intérêt les raisons pour lesquelles cette situation s’est déclenchée et ainsi enrichir une fois de plus leur système de croyances.

En revanche, lorsque l’individu estime que l’action est un vecteur ou une entrave suffisante à l’atteinte de ses désirs, celui-ci peut choisir de s’intégrer à l’action collective. De là, il essaiera par ses moyens, d’amoindrir, d’amplifier, d’annuler ou encore de faire dévier l’action en cours.

De ce fait, lors d’une action individuelle menant à une action collective, certains acteurs n’interviendront pas, tandis que d’autres interviendront par une prise de parti active.

On peut donc juger de l’ampleur d’une action collective par le nombre de personnes prenant part à cette action, mais aussi par les moyens qu’ont ces personnes pour faire pencher la victoire dans le camp souhaité. Les témoins passifs peuvent tout de même être pris en compte dans l’ampleur du phénomène car ils sont d’une part affectés par les évènements et d’autre part agissent par leur inaction.

On remarque aussi qu’un acteur passif n’agit que dans sa propre sphère individuelle et non pour l'intérêt collectif. Quant à l’action active, elle est efficiente à la fois dans la sphère individuelle et collective. Les réflexions qu’impliquent les actions passives ne sont pas toujours réalisées en vue d’une action active ultérieure, puisque l’individu peut choisir de rester systématiquement passif vis-à-vis d’une même action qui se répète ou de n'en tirer aucun enseignement. Aussi, l'action passive d’un individu reste tout de même une action active vue par les individus dans l’action, puisque refuser de prendre part à l'action en cours est une forme de contestation réduisant les effectifs nécessaires à la réalisation du projet en cours. Une action passive n’est donc réellement passive que pour l’individu qui l’exécute et résulte d’un simple choix de ne pas faire pencher la balance vis-à-vis de l’action qu’il observe.

Aucun phénomène collectif n’émerge directement, c'est le fruit d’une action en premier lieu individuelle et dont le résultat aboutit à la prise de parti d’un groupe de personnes jugeant du bien-fondé de cette action de manière personnelle et collective. Il en résulte tout de même que chaque individu, bien qu’affecté par les actions du groupe, effectue lui-même le choix personnel d’enrichir ou non en effectifs et en moyens l’action en cours.

Bien sûr, dans certains cas, l’individu peut être menacé s’il ne prend pas part à l’action, mais aura toujours le choix de subir les châtiments dictés par les menaces plutôt que de prendre part au projet. Dans le cas d’une guerre, les menaces sont cependant suffisamment importantes pour empêcher en grande partie la désertion.

\begin{center}
***
\end{center}

De ce constat, une action collective est le résultat d’un ensemble d’actions déclenchées par une action individuelle unique. Toute action individuelle ou collective a lieu par l'influence d'action(s) antérieure(s) effectuée(s) par des personnes tierces, hormis les actions destinées à sustenter nos besoins vitaux ou visant à l'auto-satisfaction. Ce serait donc un chaos total sans communication adaptée, une dimension éthique commune entre les individus exécutant l’action, et des règles adéquates et préalables à respecter.

La règle unique : « attrapez ce chat » pourrait prendre un aspect risible si aucun des trois prérequis cités plus haut n’étaient respectés. La dimension éthique dira par exemple si on a la possibilité de tuer le chat en l’attrapant ou de lui briser une patte. Les règles préalables éviteront des actions en désaccord et inharmonieuses entre les participants, tandis qu’une communication efficace permettra des mouvements plus fluides dans la réalisation des actions permettant l'aboutissement du projet.

Que ce soit pour le cas des guerres présenté plus haut ou pour un autre dessein, sans grammaire régissant les actions collectives, toutes les actions humaines collectives comme individuelles deviendraient inharmonieuses et inefficaces. C'est pourquoi une importance capitale est donnée à la discipline militaire, toute défaite ayant dans ce cas des conséquences dramatiques.

L'éthique commune, l'ensemble de règles communes ou particulières, une communication adaptée, et une hiérarchie pertinente sont donc les quatre piliers pour une organisation cohérente menant à la réussite de n'importe quel projet collectif.

Dans le livre I, nous avons mis en avant une source d'énergie originelle, puis une mise en mouvement originelle. Mais par ce raisonnement, on peut aussi admettre une action consciente originelle ayant inspiré toutes les autres. Qui en est responsable et quand la dater ? N'est-ce pas avant tout le résultat de l’œuvre de l'être suprême à l'origine de la création?

\begin{center}
***
\end{center}

Pour ces raisons, dans le cas d'une civilisation saine, composée d’un nombre conséquent d’individus, il y a nécessité de créer un système de lois morales et juridiques évitant au mieux notamment les actions collectives et individuelles violentes, et permettant une réalisation facilitée des desseins estimés comme essentiels à cette société, le tout permettant de nettement amoindrir toutes les actions représentant un danger au niveau collectif afin d'assurer la pérennité de cette civilisation.

Ces éléments, une fois présents, doivent (s’ils sont habilement conçus) permettre une prompte réalisation des desseins multiples d’un individu, d’un groupe d’individus, d’une société, d’une civilisation en faveur du groupement humain composé.

Pour résumer, les actions personnelles ou communes doivent favoriser le développement des sociétés au même titre que la société favorise la survie et le bien-être des citoyens qui la composent. Tous ces desseins additionnés (si leurs entreprises ont réussi) alimentent donc un système qui les alimente en retour.

En revanche, il n’arrive certainement jamais que tous les individus de ce système partagent à parts égales les fruits de l’aboutissement des actions collectives.

Malgré tout, il y a nécessité de déterminer la grammaire (ou l'ensemble de règles) la plus adaptée possible pour qu’un individu ou un groupe parviennent à leurs fins. Le choix du terme "grammaire" prend appui dans la théologie où il est possible de voir ces actions comme des imitations imparfaites d’un langage pouvant être considéré comme parfait.

Il y aurait alors potentiellement existence mais méconnaissance de la grammaire (donc des règles) la plus adaptée aux desseins des hommes, quel que soit le lieu et l’époque. Si cette grammaire était appliquée, elle le serait de manière imparfaite du fait que chaque homme est sujet à l'erreur, mais aussi à des volontés de transgression de par sa nature pécheresse.

Cette grammaire constitue alors les lois d'un système religieux, bien que parfois, comme pour le christianisme ou dans l'athéisme occidental, les lois civiques ne soient pas considérées comme des questions religieuses. Pour autant, dans les deux cas, ces lois civiques sont alignées à une éthique, donc il y a encore religiosité.

En revanche, dans le cas où cette grammaire est inapte à la réalisation des desseins humains ou finit par le devenir, la civilisation ou le groupe est anéanti, et l’on peut citer le passage de la Torah sur le veau d’or lors de l'exode du peuple juif.


\begin{center}
\begin{quote}
\textit{Yahvé dit alors à Moïse: "Va, redescends! Car ton peuple s'est corrompu, ce peuple que tu as fait sortir du pays d'\'Egypte! Ils n'ont pas été longs à se détourner de la voie que je leur avais enseignée: ils se sont fait un veau de métal fondu, ils se sont prosternés devant lui et lui ont présenté des sacrifices. Ils ont même dit: Israël, voici tes dieux qui t'ont fait sortir du pays d'\'Egypte!" \\ Yahvé dit encore à Moïse: "j'ai bien compris que ce peuple a la nuque raide. Laisse moi donc déchaîner contre eux ma colère: je vais les faire disparaître, mais je ferai sortir de toi une grande nation."}
\end{quote}
Exode, chapitre 32 verset 7 à 10.
\end{center}

Bien sûr, cet extrait de la Torah reste, selon moi, à interpréter d'un point de vue allégorique, comme une mise en garde contre l'égarement religieux, l'adoption de fausses croyances ou de règles néfastes qui mènent à leur perte ceux qui y adhèrent. Leurs actions individuelles et collectives deviennent alors inadaptées à la réalisation de leurs desseins, voire à leur survie. Je trouve ce passage riche d'enseignements.


\chapter{Opinion sur les prophètes et les écrits religieux}

De mon point de vue, les prophètes existent sans exister et il n'y a en réalité que des fous qui ont été écoutés. Pour autant, mon doute sur l'existence réelle des prophètes subsiste, car il reste délicat de rendre rationnels les miracles qu'ils ont accomplis, au-delà des propos qu'ils ont tenus.

Néanmoins, certains auraient pu être atteints de folie géniale, et leurs actions individuelles ayant mené à des actions collectives auraient tout autant été marquées par cette folie.

Les Grecs parlaient d'ailleurs de folie divine, là où, à notre époque, tout type de folie est perçu comme infructueux et reste systématiquement condamnable.

Mais il est tout à fait possible que certains de ces "fous" détiennent ou ont détenu des connaissances majeures sur le divin.

En assemblant les connaissances de ces individus, il est alors possible de construire un recueil composé de l'intégralité des connaissances humaines sur Dieu.

Sans doute ce recueil resterait-il très incomplet. D'ailleurs, avons-nous au moins un livre religieux réellement satisfaisant ?

Il reste encore que cette entreprise de construction de canon religieux est périlleuse, incertaine, sujette à l'erreur (n'en déplaise aux musulmans), divise et va parfois jusqu'à détruire des peuples.

\chapter{Solidité des dix commandements face à ceux de différents cultes}

Je pense que les dix commandements sont à la fois trop élémentaires et trop pertinents pour qu’il n’y ait pas au moins une part de divin en eux. Ces commandements constituent aussi, selon moi, le noyau des civilisations les plus saines.

Dans son ouvrage "Le Génie du Christianisme", Chateaubriand compare les différents équivalents aux dix commandements appliqués dans d’autres civilisations. Citons-en quelques-uns:

\vfill
\newpage
\textbf{Lois des Gaules ou des Druides.}

\begin{quote}
\begin{itemize}

\item "l’univers est éternel, l’âme immortelle.
\item Honore la nature.
\item Défendez votre mère, votre patrie, la terre.
\item Admets la femme dans tes conseils.
\item Honore l’étranger, et mets à part sa portion dans ta récolte.
\item Que l’infâme soit enseveli sous la boue.
\item N’élève point de temple, et ne confie l’histoire du passé qu’à ta mémoire.
\item Homme, tu es libre : sois sans propriété.
\item Honore le vieillard, et que le jeune homme ne puisse déposer contre lui.
\item Le brave sera récompensé après la mort et le lâche puni."

\end{itemize}
\end{quote}

\vfill
\newpage
\textbf{Lois Égyptiennes.}

\begin{quote}
\begin{itemize}
\item "Cnef, dieu universel, ténèbres inconnues, obscurité impénétrable.
\item Osiris est le dieu bon ; Typhon le dieu méchant.
\item Honore tes parents.
\item Suis la profession de ton père.
\item Sois vertueux ; les juges du lac prononceront après ta mort
sur tes œuvres.
\item Lave ton corps deux fois par jour, et deux fois la nuit.
\item Vis de peu.
\item Ne révèle point les mystères"
\end{itemize}
\end{quote}

\vfill
\newpage
\textbf{Lois de Pythagore.}

\begin{quote}
\begin{itemize}
\item "Honore les dieux immortels, tels qu’ils sont établis par la loi.
\item Honore tes parents.
\item Fais ce qui n’affligera pas ta mémoire.
\item N’admets point le sommeil dans tes yeux, avant d’avoir examiné trois fois dans ton âme les œuvres de ta journée. Demande-toi : Où ai-je été ? Qu’ai-je fait ? Qu’aurais-je du faire ?
\item Ainsi, après une vie sainte, lorsque ton corps retournera aux éléments, tu deviendras immortel et incorruptible : tu ne pourras plus mourir."
\end{itemize}
\end{quote}

\vfill
\newpage
\textbf{Les dix commandements (traduction de Louis Segond).}
\begin{quote}
\begin{itemize}
\item "Tu n’aura point d’autres dieux devant ma face.
\item Tu ne feras point d’image taillée, ni de représentation quelconque des choses qui sont en haut des cieux, qui sont en bas de la terre, et qui sont dans les eaux plus bas que la terre. Tu ne te prosternera point devant elles, et tu ne les serviras point; car moi, L’Éternel, ton Dieu, je suis un Dieu jaloux, qui punis l’iniquité des pères jusqu’à la troisième et à la quatrième génération de ceux qui me haïssent, et qui fait miséricorde jusqu’à la millième génération à ceux qui
m’aiment et qui gardent mes commandements. 
\item Tu ne prendra point le nom de L’Éternel, ton Dieu, en vain; car L’Éternel ne laissera point impuni celui qui prendra son nom en vain.
\item souviens-toi du jour du repos pour le sanctifier. tu travailleras six jours, et tu feras tout ton ouvrage. Mais le septième jour est le jour de repos de L’Éternel, ton Dieu : tu ne feras aucun ouvrage, ni toi, ni ton fils, ni ta fille, ni ton serviteur, ni ta servante, ni ton bétail, ni l’étranger qui est dans tes portes. Car en six jours L’Éternel a fait les cieux, la terre, la mer et tout ce qui y est contenu, et il s’est reposé le septième jour : c’est pourquoi L’Éternel a béni le jour du repos, et l’a sanctifié.
\item Honore ton père et ta mère, afin que tes jours se prolongent dans le pays que L’Éternel, ton Dieu, te donne.
\item Tu ne tueras point.
\item tu ne commettras point d’adultère.
\item Tu ne déroberas point.
\item Tu ne porteras point de faux témoignage contre ton prochain.
\item Tu ne convoiteras point la maison de ton prochain; tu ne convoiteras point la femme de ton prochain, ni son serviteur, ni sa servante, ni son bœuf, ni son âne, ni aucune chose qui appartienne à ton prochain."
\end{itemize}
\end{quote}

\begin{center}
***
\end{center}


Certaines religions n’admettent pas l’existence d’un libre arbitre et laissent penser que nous vivons dans l’illusion que nous exécutons nos choix alors que tout notre vécu est fixé d’avance par notre destinée. Mais je crois personnellement en l’existence du libre arbitre. Les Grecs et les Romains croyaient au destin, les Vikings tout autant, et les protestants calvinistes pensent que nous sommes des robots exécutant des actions tout à fait déterministes au regard de Dieu.

Pourtant, l’omniscience divine prête à penser que nos choix sont parfaitement connus au moins de notre Dieu; mais ce Dieu est aussi omnipotent, illimité en capacité créatrice, et peut donc très bien créer une pierre trop lourde à porter pour lui-même. Dieu peut donc tout simplement s’interdire de connaître une partie ou même l’ensemble des choix que nous effectuons.

Partant du principe que nous effectuons des choix individuels et collectifs, certains sont bénéfiques et d’autres mènent à des difficultés et des chaos dans les cas extrêmes. Si les choix bénéfiques sont des grâces divines et les choix néfastes des châtiments, je pense qu’il est judicieux de s’orienter vers des choix apportant des bénédictions, car d’une part, cela est plus confortable lors de notre vie terrestre, et d’autre part, potentiellement mieux rétribué une fois notre passage sur terre arrivé à son terme.

Il n’est cependant pas évident de mesurer les effets de ces choix, dont certains pouvant paraître bénéfiques dans l’immédiat se transformeront en calamité à moyen ou à long terme. Je fais donc le choix personnel de respecter au mieux les dix commandements, car on a la preuve par l’expérience (différentes civilisations ayant adopté d’autres modèles et servant de comparatif) que rien n’est plus profitable au développement humain dans le cas où ces commandements constituent le noyau éthique et comportemental d’une civilisation. L’application des dix commandements est donc nettement porteuse de bénédictions à court, long et moyen terme, même s’il peut être, dans certains cas, douloureux de s’imposer cette discipline. Les dix commandements assurent, à mon sens, plus de vitalité et de pérennité à n’importe quelle civilisation, quel que soit son climat et sa situation géographique, que n’importe quel autre équivalent. Il suffit d'observer le prestige civilisationnel des sociétés qui les ont appliqués (les Romains et les Grecs ayant été dépassés, leurs empires ayant fini par chuter).

Les lois de nos sociétés interdisent toujours le meurtre et le vol, c’est sûrement de bonne augure, mais qu’en est-il du reste ?

Que les hommes le veuillent ou non, tout individu ainsi que toutes ses créations matérielles et abstraites, et par extension les lois humaines, sont soumises aux lois divines et, comme je l’ai évoqué dans le chapitre 12, les déclins arrivent lorsque l’on s’éloigne trop dangereusement de ces lois inébranlables.

Pour ma part, ayant un point de vue proche de celui d’un chrétien à ce sujet, qu’il y ait ou non des lois poussant les individus à appliquer ces commandements m’importe peu. Je perçois ces commandements comme des lois morales que l’on décide ou non de s’imposer. Pour ce qui est du système juridique, citons simplement les propos du Christ : \begin{quote}\textit{"Rendez à César ce qui appartient à César et à Dieu ce qui appartient à Dieu."}\end{quote}

Pour digresser, cette parabole constitue une différence majeure qu'a le christianisme avec l'islam, qui impose une loi civique, tandis que le christianisme ne vient seulement que prescrire des lois morales et laisse le législatif à ceux qui détiennent le pouvoir. À mon sens, c'est une des raisons de la supériorité de cette religion en termes de prestige civilisationnel et de son aptitude à évoluer selon les époques qu'elle traverse depuis plus de deux mille ans.

Aucune preuve ontologique ne peut réellement appuyer ces propos. Je présume que le bon sens des dix commandements pourra être vérifié par chacun au fil des évolutions de nos sociétés et des enseignements tirés des cataclysmes futurs. Affaire à suivre donc...

\part{Sur la négligence et le dévouement à Dieu}

\chapter{La grande usurpation savante}

Il serait bon de stopper cette trop forte distinction entre sciences dites "dures" et sciences dites "molles", surtout en ce qui concerne la philosophie. D'ailleurs, des philosophes tels qu'Emmanuel Kant écrivaient la philosophie comme des mathématiques pures. Il serait judicieux de rappeler les bouleversements scientifiques et paradigmatiques de la philosophie comme celles des religions.

Le christianisme a, par exemple, éliminé les croyances aux naïades, aux nymphes, etc., et très tôt dans sa longue histoire, à la plupart sinon toutes les créatures mystiques inexistantes. De ce fait, grâce à cet apport religieux, la nature a grandement perdu de son mystère et par conséquent, une partie importante de son mysticisme.

Toutes ces évolutions, scientifiques, philosophiques et théologiques ont globalement cheminé vers plus de justesse (comme évoqué dans le chapitre 8). Malheureusement, à l'époque actuelle, nos élites savantes semblent majoritairement enseigner que religion et philosophie sont au mieux surannées, sinon mortes.


\begin{center}
\begin{quote}
\textit{"Dieu est mort."}\end{quote} Friedrich Nietzsche.
\end{center}



\begin{center}
\begin{quote}
\textit{"La philosophie est morte."}\end{quote} Stephen Hawking.

\end{center}


\'Eliminer la spiritualité par et pour le spirituel afin d'atteindre plus de profondeur métaphysique ?!

Puisqu'on cherche à enseigner que la philosophie est morte ou, au mieux, tombée en désuétude, alors que cette assertion résulte d'une ébauche maladroite de raisonnement philosophique (puisque non prouvé mathématiquement), j'attends de la part des élites qui diffusent cette idée la démonstration mathématique de la soi-disant mort de la philosophie, en leur souhaitant un prompt succès !

Bien sûr, et heureusement, cette idée n'est pas partagée par l'ensemble des doctes contemporains.

La physique, le calcul mathématique, formel ou numérique, sont des sciences qui ne servent pas à elles-mêmes et, nous l'avons vu, n'ont pas d'autres applications concrètes que la construction de machines, de ponts, de routes etc... (cf. chapitre 8).

Ces assertions contemporaines n'ont que pour effet de discréditer et amoindrir la portée des propos des philosophes et des théologiens qui, par leur volonté de compréhension du monde, sont et resteront, pour moi, les artisans de premier ordre des avancées humaines des degrés les plus élevés. Avancées qui ne doivent d'ailleurs pas rester purement techniques.

La philosophie est-elle morte ? Laissons alors entre les mains de nos mécaniciens le soin de se charger, grâce à leurs connaissances techniques, des questions éthiques, politiques, métaphysiques... Je me permets aussi de leur rappeler que tous les principes de la physique classique encore en vigueur sont eux-mêmes enracinés dans la philosophie naturelle, elle-même enracinée dans l'ontologie biblique, et n'ont pas de démonstration autre (cf. chapitre 2).

Laissons-les donc enliser nos civilisations dans un désastre spirituel total afin de permettre à tout un chacun de constater, une bonne fois pour toutes, le péril de la démarche par le déclin.

Demandons aussi aux historiens compétents si l'expérience n'a pas déjà été tentée. Ce n'est certainement pas la première fois que l'homme essaye en vain de défier ou d'abandonner Dieu, avec les conséquences funestes chroniquées dans les écrits religieux et notamment bibliques. N'est-il pas stupide de répéter sans arrêt les mêmes erreurs ?


\begin{center}
***
\end{center}

Selon Aristote, la hiérarchie des sciences théoriques est établie selon leur degré de généralité et d'abstraction, et elles sont classées de la manière suivante :

\begin{itemize}
\item La métaphysique : de plus haut degré, étudiant l'être en tant qu'être, elle explore les concepts les plus généraux et abstraits.
\item La physique : étudie la nature et les principes fondamentaux qui gouvernent la matière.
\item Les mathématiques : de dernier degré, traitant des quantités dénombrables et des relations abstraites.
\end{itemize}

Ainsi, la physique et les mathématiques se trouvent à des degrés inférieurs à la métaphysique, qui est avant tout l'apanage des philosophes et des théologiens. Physique et mathématiques sont contraints de prendre racine dans la métaphysique.

\begin{center}
***
\end{center}

Que nos spécialistes des mouvements de la matière m'expliquent enfin la cause de ces mouvements par la preuve mathématique ou physique, alors que pour l'instant, le mouvement reste encore un constat empirique. Qu'ils répondent enfin à la question et apportent la preuve que tous les peuples se sont religieusement posée : avons-nous réellement un libre arbitre ou effectuons-nous des actions totalement déterministes ? L'univers est-il borné ou infini et, s'il est borné, qu'y a-t-il de l'autre côté de ses limites, puisque nous avons vu que le néant est un concept qui ne peut exister, puisqu'un concept n'est pas du néant ? Le monde que l'on observe existe-t-il concrètement ou est-ce une illusion ?

Ces questions n'ont jusqu'à présent été traitées de manières diverses au cours des siècles que par des philosophes et des prêtres. Les réponses apportées sont encore admises par les physiciens et mathématiciens dans la tradition de notre pensée religieuse helléno-chrétienne. Je suis sidéré de voir à quel point la plupart des scientifiques contemporains l'ignorent et admettent ces dogmes comme s'ils étaient des choses innées ! Or pas de réponse philosophique certaine, et encore moins physique, n'est réellement établie. Seule reste la croyance religieuse pour y répondre, ce qui implique aussi la possibilité d'erreurs.

\begin{center}
***
\end{center} 

Je ne prétends heureusement pas connaître le sens de la vie, mais je ne peux pas croire qu'elle n'en possède aucun. Si la vie n'avait aucun sens, elle n'aurait aucune raison d'exister (extrapolation du principe de moindre action du chapitre 5).

Dans nos sociétés contemporaines, avec l'émergence de l'humanisme et de l'athéisme occidental, le physicien et le mathématicien ont pris la place des doctes chargés de nous enseigner une ontologie, une métaphysique, et même une religion qui n'offre aucun sens à l'existence. Ces disciplines scientifiques doivent pourtant leur établissement et leurs lois fondamentales à une ontologie, et l'ontologie doit ses fondements à une religion.

Erreurs après erreurs, victoires après victoires, les religions et leurs ontologies ont évolué et ont aussi été bouleversées ; elles ont certainement gagné en vérité, mais restent imparfaites car créées par l'homme. Les évolutions et victoires de la métaphysique, de la physique et des mathématiques en sont une conséquence.

Vous l'aurez compris, pour ma part, les "sciences dures" ont usurpé de manière illégitime et dangereuse des rôles qui ne les concernent en rien et cherchent à remplacer une discipline dont elles dépendent et à laquelle elles doivent leur existence. 

\begin{center}
\begin{quote}
\textit{"Un peu de science éloigne de Dieu, beaucoup de science y ramène."}\end{quote} Francis Bacon.
\end{center}

\chapter{Sur l’application de la théorie du surhomme dans les sociétés contemporaines}

Suite à l’engouement porté à la théorie évolutionniste de Darwin ainsi qu’aux valeurs athées apportées dans toute l’Europe par la Révolution française, Nietzsche propose alors un dogme destiné à remplacer la religion chrétienne. Considéré par certains comme un prophète à l'origine de l'ère industrielle, ce philosophe incite ses lecteurs à adopter un système de valeurs basé sur l’athéisme, où l’individu et ses propres désirs deviennent alors son unique référentiel.

Influençant grandement le régime nazi, qui, contrairement à nos sociétés modernes, se base sur une interprétation racialiste et antisémite de ces écrits philosophiques, Nietzsche influencera aussi le mouvement humaniste animant aujourd’hui, en partie, les valeurs éthiques et morales des sociétés occidentales contemporaines.

Il est à noter que Nietzsche possédait d'abord une formation de philologue et donc une connaissance très élaborée des textes anciens, notamment religieux, dont sa philosophie est inspirée.

Le système de mœurs qui, selon lui, serait le plus judicieux à adopter est principalement inspiré de celui employé par les anciens peuples scandinaves de l’Europe du Nord. Par comparaison avec les sociétés occidentales actuelles, on peut mettre en évidence certains éléments tels que le débridement sexuel, ses valeurs féministes, sa tendance à l’individualisme, et aussi une certaine orientation idéologique prônant la "mort aux faibles". Il est tout de même important de souligner que les anciens peuples d’Europe du Nord prenaient soin des malades et de leurs infirmes en leur assurant une vie décente.

Pourtant, les valeurs morales et les mœurs de nos sociétés semblent souvent perçues comme bienfaisantes par une majorité et proviendraient d’une vision libérale nous menant vers un monde meilleur. Sans aller jusqu’à exiger que cette majorité lise des écrits philosophiques ni même que ces personnes comprennent Nietzsche (qui est tout de même considéré comme difficile, voire impossible à saisir),

Ces écrits tiennent plus de dogmes destinés aux élites et appliqués aux peuples occidentaux de manière voilée.

Je pense qu'il serait tout de même urgent pour ces personnes de considérer que l’on ne sort ni mœurs ni éthique d’un simple chapeau ! Si l’être humain commettait un jour la stupidité de faire le bien en ne se référant qu’à ce que ses pulsions lui dictent, nous signerions certainement l’éradication de notre espèce... Cela peut notamment s'observer par les dérives actuelles du mouvement "woke", qui, se voulant bien-pensant, mène aussi à des formes extrémistes dangereuses.

\begin{center}
\begin{quote}
\textit{"L'enfer est pavé de bonnes intentions."}\end{quote} Bernard de Clairvaux.
\end{center}

Le dogme nietzschéen est avant tout une amplification de celui apporté par la mythologie nordique et provient aussi d'une étude approfondie et philologique des phénomènes sociaux liés au déclin civilisationnel de différents peuples. Il est à noter que le paganisme des pays nordiques était à l’époque perçu comme satanique par l’Église catholique, puisque totalement contraire aux standards des valeurs chrétiennes les plus fondamentales.

Il est aussi possible d'effectuer un parallèle avec les évolutions rapides de nos sociétés modernes et celles des Vikings, qui ajoutaient tous les siècles environ, une nouvelle divinité à leur panthéon, et donc de nouvelles règles à suivre.

J’ai la forte appréhension que l’application de la théorie du surhomme générerait un déclin sans précédent si nos dirigeants devenaient réellement immodérés dans l’application de ces écrits. Fort heureusement, son œuvre est interprétée de façons très diverses et, je l'espère, rarement de manière aussi alarmante.

Pour ma part, je ne vois pas Nietzsche comme un véritable athée (ce serait mal le comprendre) mais plutôt comme un agnostique fou furieux dont l’œuvre, si son application devenait drastique, permettrait de vérifier par l'expérience sociale, l’existence du Dieu de l’Ancien et du Nouveau Testament, mais aussi de la validité de la théorie évolutionniste.

L’aboutissement de la théorie du surhomme, telle qu’elle est proposée dans l’ouvrage "L'Antéchrist", serait l’adoption de mœurs purement satanistes basées sur une inversion totale de celles du christianisme. L’homme s’infligerait alors par lui-même les mœurs les plus inadaptées à sa survie que l’on puisse imaginer, et son environnement deviendrait alors le plus rigoureux et le plus meurtrier possible.

La conclusion s’observe donc directement dans le titre de son ouvrage. Appliquer totalement la théorie du surhomme aurait alors deux effets possibles : soit, dans le cas où le Dieu biblique existe, nous pourrions observer l’arrivée de l’Antéchrist et la réalisation des prophéties eschatologiques de l’Ancien et du Nouveau Testament ; soit, dans le cas où ce Dieu n’existe pas et que la théorie évolutionniste est valide, accéder à une post-humanité plus puissante par la sélection naturelle, et donc au fameux surhomme de la théorie sensée l'engendrer.

\begin{center}
\begin{quote}
\textit{"Qu’est le singe pour l’homme ? Une dérision ou une honte douloureuse. Et c’est ce que doit être l’homme pour le surhomme : une dérision ou une honte douloureuse."}\end{quote} Friedrich Nietzsche, \\ ainsi parlait Zarathoustra.
\end{center}

Par cette interprétation, il est un peu mieux possible de constater l’immense génie pervers de Nietzsche, potentiellement cause de la démence qu’il a contractée. Par son apport, nous vivons désormais comme les cobayes d'une expérimentation funeste. Fort heureusement, toutes nos élites ne semblent pas s'attacher à ses considérations. Affaire à suivre...


\chapter{Négation de l’idée la valeur humaine}

Comme pour tout cet ouvrage, cette considération n’engage que moi et ceux qui pensent de cette manière. J’estime d’ailleurs avoir développé cette idée trop tardivement au cours de ma vie.

Je rejoins tout à fait l’opinion de Montaigne selon laquelle il n’y a pas de grands hommes mais simplement des hommes. L’idée que l’on peut se faire de la grandeur humaine se construit à partir de qualités désirables qui dépendent, dans une certaine mesure, de la considération de chaque individu. Cette idée de grandeur d’âme peut alors se développer par rapport à certains critères tels que la vertu, la quantité d’argent, le pouvoir, les capacités intellectuelles ou la force physique. Certaines de ces qualités peuvent être perçues comme perverses ou non en fonction des individus, de leurs croyances et de leurs cultures.

À mon goût, il y aura toujours trop de personnes s’adonnant à ces jugements de valeurs. Mais pourquoi émettre moi-même un jugement sur ces personnes ? Je souhaite quand même rester au mieux conforme à cette idée.

De mon point de vue, tout individu possède un parcours de vie digne d’intérêt qui lui est propre. S'il y a un sens à la vie de manière absolue, il est connu de Dieu seul. Dans ce cas, seul le fait de trouver et d’atteindre les objets de ses désirs pourrait compter, et la nature humaine est telle que nous resterons tous éternellement insatisfaits.
\begin{center}
\begin{quote}
\textit{"Il n'y a de plaisir que dans le boire et le manger."}
\end{quote} Livre de l'\'Ecclésiaste
\end{center}

Ces intérêts peuvent totalement diverger d’une quelconque volonté de puissance. Tout le monde s’est déjà brûlé en persistant dans cette voie. Comment pouvons-nous savoir objectivement ce qui est louable en traitant des données tout à fait subjectives ?
Que ces individus aient la santé, des richesses, de l'intelligence ou de la gloire
n’y change rien.

Je trouve que tout parcours de vie a une valeur initiatique et qu'il n'y aurait pas de réelles inégalités de destins, seulement des parcours de vie différents de valeurs équivalentes. Je vois la magnanimité, la puissance et la grandeur comme des illusions collectives servant d'œillères à la vanité de notre existence.

Pour ma part, je m’intéresse très peu à l’argent et au pouvoir. Je désire surtout une santé que je ne peux malheureusement pas atteindre pleinement et mon élévation spirituelle. Le simple fait de me procurer des livres ou de m'adonner à des activités intellectuelles est un vrai levier d’action me permettant d’évoluer dans cette voie.

Ces désirs n’engagent que moi et ceux qui pourraient en être intéressés. Avancer, même assez lentement, sur ce chemin n’est ni plus ni moins digne que d’avancer vers autre chose. À chacun sa vanité, et voici la mienne.

J’ai tout de même en dégoût les individus pervers prenant plaisir au mal et se glorifiant de pouvoir faire souffrir. L’essentiel pour moi reste alors d’éviter au mieux ces comportements et d’essayer de me corriger lorsqu’il m’arrive d’emprunter ce sentier.


\chapter{Conclusion: mon testament, pacte avec Dieu}

Peu importe la cause de mon décès, il reste néanmoins possible qu’il soit dû au déclenchement d’une guerre. Dans ce cas, je donnerai volontiers ma vie pour préserver certaines de mes vertus et éviter mon avilissement.

Quoi qu’il puisse arriver, quel que soit le contexte dans lequel je serai projeté, je respecterai toujours au mieux les dix commandements (cf. traduction de Louis Segond au chapitre 13).

De par ma nature pacifique, les commandements que je suis sûr de respecter au mieux sont ceux qui prescrivent de ne pas tuer et de ne pas voler.

Il m’est arrivé de mentir, de commettre des adultères, d’en vouloir à Dieu, et je n’ai, de par mon éducation athée, pas toujours cru qu’il y avait un Dieu.

Mais je fais le serment de ne pas tuer, même en cas de légitime défense. En cas de guerre ou de simple confrontation avec une autre personne ou un groupe, je pourrais me défendre, mais je refuserai toujours de prendre la moindre vie. Il se peut que je cherche à fuir pour prolonger ma vie terrestre. J’ai d’autres projets que celui de mourir de manière stupide. Il est cependant hors de question de m’adonner à des actions trop basses pour prolonger ma vie. Tuer ou voler sont donc pour moi totalement exclus, même en cas d’absolue nécessité.

Pour le reste, je ferai de mon mieux pour satisfaire ces commandements. Je ne suis pas convaincu du caractère divin du Christ, mais j’ai tout de même la conviction d’un Dieu omnipotent et de l’immortalité de l’âme. Lorsque celle-ci est liée au corps et au monde terrestre, elle laisse une empreinte indélébile lors du passage à l’étape suivante qui suit la mort du corps (cf. chapitre 10).

Je n’ai pas une grande affinité avec le bouddhisme et l’hindouisme, et mes connaissances concernant ces religions sont très limitées (je ne dispose malheureusement pas d’un temps infini pour les étudier). Je crois que ces religions sont tout à fait louables et pacifiques, mais je ne crois pas en la réincarnation.

Le peuple grec, que j’ai pu étudier à travers certains de leurs écrits philosophiques, y croyait. Socrate a cependant eu, par la suite, une influence considérable vers l’idée d’un Dieu unique, omnipotent et démiurge de l’univers, qui est, selon lui, celui qui a été, est, et sera (pouvant se traduire en hébreu par le nom de Jéhovah). C’est après avoir été accusé de corrompre la jeunesse, par les croyances en d'autres divinités qu’il diffusait de manière orale, qu’il fut condamné à mort. La chronique de son procès, rédigée par Platon, est toujours étudiée en faculté de droit.

Pour en revenir à mon pacte avec le divin, je ne laisserai jamais l’empreinte sur mon âme d’un meurtre commis sur le royaume terrestre.

\begin{center}
***
\end{center}

Aristote nous dit dans "Les Politiques" que la tyrannie peut exister dans n’importe quel type de régime. Il est probable que tout régime, bon ou mauvais, possède toujours une pointe de tyrannie, même si nous vivions dans le meilleur des mondes. Même si parfois, le mal s’élève et s’étend dans des proportions alarmantes au sein des communautés humaines, avec pour exemple marquant le régime nazi, ces phénomènes sont tout à fait cycliques, et le bien et l’équité finissent toujours par se restaurer par la force des choses ou par l’apparition d’un individu hors du commun, parfois élevé au rang de prophète ou de Messie.

La traduction du mot grec "apocalypse" est "révélation", et des déclins les plus funestes renaissent de nouvelles lois, de nouvelles mœurs, une nouvelle éthique. En clair, une nouvelle religion ou un enrichissement de l’ancienne religion plus adapté, où l’homme a su tirer les conséquences de ses erreurs passées.

Personne ne peut anticiper les catastrophes que l’humanité traversera, mais les périodes les plus meurtrières ont la particularité de cumuler quatre fléaux pouvant être assimilés aux cavaliers de l'apocalypse (Mort, Famine, Guerre et Conquête) soit:

\begin{itemize}
\item la guerre,
\item la famine,
\item les maladies,
\item les bêtes sauvages.
\end{itemize}

Quel(s) autre(s) fléau(x) pourrait(ent) être plus dévastateur(s) ?

Dans le cas de guerre, de maladies et de famine trop importantes, l’homme n’a plus les moyens logistiques pour dominer les animaux de la terre, et c’est à ce moment-là que les bêtes sauvages peuvent représenter un fléau important. Ces événements appartiennent à la nature du monde et poussent l’homme à l’avilissement, autant que l’avilissement des hommes alimente ces fléaux. Dans le cadre de l’eschatologie juive, chrétienne et musulmane, une apocalypse finale, sans précédent, où absolument tout est révélé à l’homme, porte avec elle un millénaire de paix où la science, la technique et le bien-être humain n'ont encore jamais été égalés.

Je crois possible cette apocalypse. Constatons-nous alors, une fois tout ce malheur terrestre passé, que les dix commandements sont totalement fondamentaux ? Trouverons-nous des commandements encore plus divins, plus adaptés, et encore plus porteurs de vitalité et de prestige civilisationnel ? Le christianisme sera-t-il triomphant ?

J’ai tout de même la nette impression que l’humanité ne trouvera pas mieux que les dix commandements de l’Ancien Testament, car des commandements meilleurs ne peuvent pas exister, à mon sens.

Que je voie ces fléaux arriver de mon vivant ou non, et quelle que soit la cause de la mort de mon corps, je souhaite avoir l’opportunité de laisser à mon âme une belle empreinte de mon passage terrestre. À l’instar des fléaux qui accentuent le péché et des péchés qui alimentent les fléaux, mon passage terrestre laissera une empreinte sur mon âme, tout comme mon âme laissera une empreinte sur le monde matériel (cf. chapitre 10). Mais une empreinte peu marquante, autant qu’une empreinte durable sur le matériel, finit toujours par s’éteindre de par la nature corruptible du monde matériel. La durabilité de l’empreinte que je laisserai au matériel m’importe peu. De ceux qui ont laissé une forte empreinte, nous ne retenons pas que les plus nobles d’entre nous, et l’histoire référence autant de monstres que de sauveurs.

Les Vikings avaient raison de croire que toute âme poursuivait ses combats sur le monde terrestre, y compris de manière post-mortem. Nous laissons inévitablement une empreinte spirituelle et matérielle sur le royaume terrestre, mais je crois que seule l’empreinte du matériel sur le spirituel est réellement éternelle.

\begin{center}
***
\end{center}

Mais ensuite, de quoi sera faite cette vie sans corps ni sensations corporelles ? Quelles sont les lois du monde immatériel ? Quelles seront alors les exigences de Dieu pour chacun de nous ? Les bénédictions ou les châtiments qui nous seront infligés seront-ils uniquement ceux induits par l’empreinte laissée par notre passage terrestre ? J’ai foi en une équité absolue des lois divines, et je suis donc convaincu que la rétribution des actes de chacun sera juste de manière tout aussi absolue.

N’ayant aucun point de vue matérialiste, l’univers est donc, pour ma part, composé de parties matérielles et immatérielles. L’homme, les bêtes, les végétaux et toute forme de vie sont composés de ces deux aspects (âme et corps). Par croyance, je pense que la partie immatérielle de la vie subsiste à la mort physique du fait de la nature incorruptible de l'immatériel.

Une vie après la mort ne serait alors pas destinée uniquement à l’homme, mais à toutes les espèces vivantes. Ce même principe d’empreinte des actions terrestres sur une âme immortelle s’applique alors à tout le vivant. Dieu doit cependant avoir des exigences diverses en fonction de ces espèces, puisque les comportements qu’elles doivent adopter pour assurer leur subsistance sont souvent totalement différents.

Il se pourrait que certaines actions menées sur terre soient abominables aux yeux de Dieu, au point qu’il décide de réduire à néant les âmes trop corrompues (c'est aussi une croyance chrétienne). Si ce Dieu est omnipotent, il en a la capacité. Mais le souhaite-t-il ? Le fera-t-il ? Nous laissera-t-il l’ultime choix de rester près de lui pour avoir la vie ou bien de s’en écarter totalement et d’infliger alors une mort définitive ?

Dans le catholicisme contemporain, il est considéré que l'enfer existe, mais qu'il n'y a pas nécessairement des âmes qui y sont damnées, puisque Dieu laisse toujours le choix, dont celui de revenir à lui. Il peut simplement être douloureux pour chacun de délaisser certains plaisirs terrestres dont nous serions trop tributaires afin d'être accueillis par Dieu à bras ouverts. Si le divin est bien édifié de la sorte, j'ai pour ambition d'y parvenir et de l'intégrer.

\end{document}
